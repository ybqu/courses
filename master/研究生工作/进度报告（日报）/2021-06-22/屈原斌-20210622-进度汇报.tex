%
% File coling2020.tex
%
% Contact: feiliu@cs.ucf.edu & liang.huang.sh@gmail.com
%% Based on the style files for COLING-2018, which were, in turn,
%% Based on the style files for COLING-2016, which were, in turn,
%% Based on the style files for COLING-2014, which were, in turn,
%% Based on the style files for ACL-2014, which were, in turn,
%% Based on the style files for ACL-2013, which were, in turn,
%% Based on the style files for ACL-2012, which were, in turn,
%% based on the style files for ACL-2011, which were, in turn, 
%% based on the style files for ACL-2010, which were, in turn, 
%% based on the style files for ACL-IJCNLP-2009, which were, in turn,
%% based on the style files for EACL-2009 and IJCNLP-2008...

%% Based on the style files for EACL 2006 by 
%%e.agirre@ehu.es or Sergi.Balari@uab.es
%% and that of ACL 08 by Joakim Nivre and Noah Smith

\documentclass[11pt]{article}
\usepackage{coling2020}
\usepackage{times}
\usepackage{url}
\usepackage{latexsym}
\usepackage{indentfirst}

\usepackage{times}
\usepackage{latexsym}
\usepackage{times}
\usepackage{soul}
\usepackage{url}
\usepackage{amsmath}
\usepackage{amsthm}
\usepackage{booktabs}
\usepackage{algorithm}
\usepackage{algorithmic}
\usepackage{amssymb}
\usepackage{longtable}
\usepackage{graphicx}
\usepackage{CJK}
\usepackage{multirow}
\usepackage{color}

%\setlength\titlebox{5cm}
\colingfinalcopy % Uncomment this line for the final submission

% You can expand the titlebox if you need extra space
% to show all the authors. Please do not make the titlebox
% smaller than 5cm (the original size); we will check this
% in the camera-ready version and ask you to change it back.


\title{2021年06月22日进度汇报}

\author{屈原斌 \\
  首都师范大学 \\
    {\tt ybqu@cnu.edu.cn}}

\date{}

\begin{document}
\begin{CJK}{UTF8}{gkai}

\maketitle
\CJKindent
%\begin{abstract}

%\end{abstract}

\section{今日进度}


\begin{itemize}
  \item [1.] 技术方案图
  \item [2.] 更新Attention方案
\end{itemize}

\section{工作详述}
\subsection{Attention方案}
\begin{itemize}
  \item 方案:
  \begin{itemize}
    \item [1.] 用一个单层双向GRU来获取作文表示
    \item [2.] 用一个单层双向GRU来获取标题表示
    \item [3.] 使用标题中的每个字表示与作文表示做点积后拼接的结果做分类
  \end{itemize}
  \item 实验结果:
  \begin{itemize}
    \item 见表1
  \end{itemize}
  \item 结论:
  \begin{itemize}
    \item 使用BERT-NSP结果最优
  \end{itemize}
\end{itemize}

% Table generated by Excel2LaTeX from sheet 'Sheet4'
\begin{table}[htbp]
  \centering
  \begin{tabular}{cccccccc}
    \hline
    \multicolumn{2}{c}{\textcolor[rgb]{ 1,  1,  1}{}} & \multicolumn{3}{c}{\textbf{优}} & \multicolumn{3}{c}{\textbf{中}} \\
    & {\textbf{Acc}} & {\textbf{Precision}} & {\textbf{Recall}} & {\textbf{F1-score}} & {\textbf{Precision}} & {\textbf{Recall}} & {\textbf{F1-score}} \\
    \hline
    {\textbf{BERT-Gen}} & 0.8814  & 0.9801  & 0.8875  & 0.9315  & 0.4216  & 0.8200  & 0.5569  \\
    \hline
    {\textbf{BERT-NSP}} & 0.9718  & 0.9929  & 0.9760  & 0.9844  & 0.7949  & 0.9300  & \textbf{0.8571 } \\
    \hline
    \textbf{Attention} & 0.9159  & 0.9739  & 0.9325  & 0.9527  & 0.5263  & 0.7500  & 0.6186  \\
    \hline
  \end{tabular}%
  \caption{实验结果}
  \label{tab:addlabel}%
\end{table}%

% \subsection{BERT分类实验结果}
% \begin{itemize}
%   \item 数据集:
%   \begin{itemize}
%     \item 乐乐课堂数据随机替换标题
%   \end{itemize}
% \end{itemize}
% Table generated by Excel2LaTeX from sheet '中文数据集'
% \begin{figure*}[htbp]\small
%   \centering
%   \includegraphics[width=1.0\linewidth]{zb.png}
%   \caption{生成样例2}
%   \label{framework}
% \end{figure*}

%\bibliography{reference}
%\bibliographystyle{coling}
%\bibliography{coling2020}

\end{CJK}
\end{document}


% include your own bib file like this:


%\begin{thebibliography}{}

%\bibitem[\protect\citename{Aho and Ullman}1972]{Aho:72}
%Alfred~V. Aho and Jeffrey~D. Ullman.
%\newblock 1972.
%\newblock {\em The Theory of Parsing, Translation and Compiling}, volume~1.
%\newblock Prentice-{Hall}, Englewood Cliffs, NJ.

%\bibitem[\protect\citename{{American Psychological Association}}1983]{APA:83}
%{American Psychological Association}.
%\newblock 1983.
%\newblock {\em Publications Manual}.
%\newblock American Psychological Association, Washington, DC.

%\bibitem[\protect\citename{{Association for Computing Machinery}}1983]{ACM:83}
%{Association for Computing Machinery}.
%\newblock 1983.
%\newblock {\em Computing Reviews}, 24(11):503--512.

%\bibitem[\protect\citename{Chandra \bgroup et al.\egroup }1981]{Chandra:81}
%Ashok~K. Chandra, Dexter~C. Kozen, and Larry~J. Stockmeyer.
%\newblock 1981.
%\newblock Alternation.
%\newblock {\em Journal of the Association for Computing Machinery},
%  28(1):114--133.

%\bibitem[\protect\citename{Gusfield}1997]{Gusfield:97}
%Dan Gusfield.
%\newblock 1997.
%\newblock {\em Algorithms on Strings, Trees and Sequences}.
%\newblock Cambridge University Press, Cambridge, UK.

%\bibitem[\protect\citename{Rasooli and Tetreault}2015]{rasooli-tetrault-2015}
%Mohammad~Sadegh Rasooli and Joel~R. Tetreault. 2015.
%\newblock {Yara parser: {A} fast and accurate dependency parser}.
%\newblock \emph{Computing Research Repository}, arXiv:1503.06733.
%\newblock Version 2.

%\bibitem[\protect\citename{Borschinger and Johnson}2011]{borsch2011}
%Benjamin Borschinger and Mark Johnson. 2011.
%\newblock A particle filter algorithm for {B}ayesian wordsegmentation.
%\newblock In \emph{Proceedings of the Australasian Language Technology Association %Workshop 2011}, pages 10--18, Canberra, Australia.

%\end{thebibliography}

