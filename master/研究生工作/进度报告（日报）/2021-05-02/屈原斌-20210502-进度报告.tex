%
% File coling2020.tex
%
% Contact: feiliu@cs.ucf.edu & liang.huang.sh@gmail.com
%% Based on the style files for COLING-2018, which were, in turn,
%% Based on the style files for COLING-2016, which were, in turn,
%% Based on the style files for COLING-2014, which were, in turn,
%% Based on the style files for ACL-2014, which were, in turn,
%% Based on the style files for ACL-2013, which were, in turn,
%% Based on the style files for ACL-2012, which were, in turn,
%% based on the style files for ACL-2011, which were, in turn, 
%% based on the style files for ACL-2010, which were, in turn, 
%% based on the style files for ACL-IJCNLP-2009, which were, in turn,
%% based on the style files for EACL-2009 and IJCNLP-2008...

%% Based on the style files for EACL 2006 by 
%%e.agirre@ehu.es or Sergi.Balari@uab.es
%% and that of ACL 08 by Joakim Nivre and Noah Smith

\documentclass[11pt]{article}
\usepackage{coling2020}
\usepackage{times}
\usepackage{url}
\usepackage{latexsym}
\usepackage{indentfirst}

\usepackage{times}
\usepackage{latexsym}
\usepackage{times}
\usepackage{soul}
\usepackage{url}
\usepackage{amsmath}
\usepackage{amsthm}
\usepackage{booktabs}
\usepackage{algorithm}
\usepackage{algorithmic}
\usepackage{amssymb}
\usepackage{longtable}
\usepackage{graphicx}
\usepackage{CJK}
\usepackage{multirow}
\usepackage{color}
\usepackage{threeparttable}

%\setlength\titlebox{5cm}
\colingfinalcopy % Uncomment this line for the final submission

% You can expand the titlebox if you need extra space
% to show all the authors. Please do not make the titlebox
% smaller than 5cm (the original size); we will check this
% in the camera-ready version and ask you to change it back.


\title{2021年05月02日进度汇报}

\author{屈原斌 \\
  首都师范大学 \\
    {\tt ybqu@cnu.edu.cn}}

\date{}

\begin{document}
\begin{CJK}{UTF8}{gkai}

\maketitle
\CJKindent
%\begin{abstract}

%\end{abstract}

\section{今日进度}


\begin{itemize}
  \item [1.] 更新离题检测实验
\end{itemize}

\section{工作详述}
\begin{itemize}
  \item 实验数据:
  \begin{itemize}
    \item ICLE数据集,11个主题,共827篇作文,离题:切题=51:776(1:15)
  \end{itemize}
  \item 实验方案:
  \begin{itemize}
    \item 聚类方案一:
    \begin{itemize}
      \item 计算聚类结果中可能离题的类与范文类的相似度,实验结果见表1
      \begin{itemize}
        \item 修改调参方式: distance\_threshold从0-1以0.05步长调参,example\_threshold大类按照比例调参(对所有类按照作文数排序,当大类包含作文数占比大于阈值时停止)
        % \item \textcolor{red}{问题:} 对于相同大小的类排序(先后顺序不同,可能会影响最终的结果)
      \end{itemize} 
      \item [-] linkage取不同参数时指标对比(doc2vec),实验结果见表2
      \item 结论:
      \begin{itemize}
        \item linkage取complete(最大值)时指标最优
        \item [?] bert生成/分类模型,取不同表示时结论不一致
      \end{itemize}
      \item 不同表示下spearman相关系数差别较大:
      \begin{itemize}
        \item 不同表示下相同主题指标最优时小类的作文数不同,会导致一些作文相似度计算方式不同(小类中的作文,取与大类距离的最小值;大类中作文,取与自己类质心的相似度)
        % \item [?] 大类和小类中的作文混在一起排序
        \item 大类和小类一起排序可能会改变原来的顺序(单独计算大小类的spearman相关系数波动较大)
      \end{itemize}
    \end{itemize}
    \item 聚类方案二(one-class):
    \begin{itemize}
      \item 计算全部作文与它们的质心的相似度,实验结果见表3
      \item 结论:
      \begin{itemize}
        \item bert生成模型指标最优
      \end{itemize}
    \end{itemize}
    \item 聚类方案三(Prompt-independent)
    \begin{itemize}
      \item 五折交叉验证,使用开发集调参,测试集测试,实验结果见表4
      \item 结论:
      \begin{itemize}
        \item 使用开发集的参数测试集测试时,部分主题没有小类(和划分范文类策略相关)
      \end{itemize}
    \end{itemize}
    \item 聚类方案四
    \begin{itemize}
      \item 计算作文与prompt的相似度,进行排序,实验结果见表5
    \end{itemize}
    \item 结论:
    \begin{itemize}
      \item bert生成模型指标最优
    \end{itemize}
  \end{itemize}
\end{itemize}

% Table generated by Excel2LaTeX from sheet 'Sheet1'
% Table generated by Excel2LaTeX from sheet '整理(余弦相似度,使用占比划分大小类)'
\begin{table}[htbp]\small
  \centering
    \begin{tabular}{c|c|c|c|c|c|c|c|c|c|c}
      \hline
      \multicolumn{2}{c|}{} & \textbf{R@10} & \textbf{R@15} & \textbf{R@20} & \textbf{R@50} & \textbf{R@all} & \textbf{P@1} & \textbf{P@5} & \textbf{P@10} & \textbf{spearman} \\
      \hline
      \multicolumn{2}{c|}{\textbf{baseline}} & 0.5334  & 0.5334  & 0.6546  & 0.7548  & 0.8225  & 0.3636  & 0.3091  & 0.2727  & 0.2007  \\
      \hline
      \multicolumn{2}{c|}{\textbf{tfidf}} & \textcolor[rgb]{ .502,  .502,  .502}{0.5009 } & \textcolor[rgb]{ .502,  .502,  .502}{0.5009 } & \textcolor[rgb]{ .502,  .502,  .502}{0.5517 } & \textcolor[rgb]{ .502,  .502,  .502}{0.6572 } & \textcolor[rgb]{ .502,  .502,  .502}{0.6840 } & \textcolor[rgb]{ .502,  .502,  .502}{0.4545 } & \textcolor[rgb]{ .502,  .502,  .502}{0.2909 } & \textcolor[rgb]{ .502,  .502,  .502}{0.2636 } & \textcolor[rgb]{ .502,  .502,  .502}{0.2148 } \\
      \hline
      \multicolumn{2}{c|}{\textbf{doc2vec}} & 0.5334  & 0.6243  & 0.6296  & 0.6888  & 0.7459  & 0.2727  & 0.2727  & 0.2727  & 0.1355  \\
      \hline
      \multirow{4}[0]{*}{\textbf{分类模型}} & \textbf{lstm} & 0.4842  & 0.5023  & 0.5259  & 0.6207  & 0.6635  & 0.4545  & 0.2545  & 0.2364  & 0.1676  \\
      & \textbf{bert\_CLS} & \textcolor[rgb]{ 1,  0,  0}{\textbf{0.5995 }} & 0.5995  & 0.6298  & 0.7064  & 0.7545  & 0.2727  & 0.2727  & 0.2727  & 0.1776  \\
      & \textbf{bert\_Last1avg} & 0.5540  & 0.6752  & 0.7358  & 0.7465  & 0.7947  & 0.3636  & 0.2727  & 0.2818  & 0.1977  \\
      & \textbf{bert\_Last2avg} & 0.5540  & 0.6752  & 0.6752  & 0.7647  & 0.8128  & 0.2727  & 0.2909  & 0.2818  & 0.1745  \\
      \hline
      \multirow{6}[0]{*}{\textbf{生成模型}} & \textbf{lstm} & 0.3781  & 0.3834  & 0.3834  & 0.4979  & 0.5460  & 0.0909  & 0.1636  & 0.2000  & 0.0792  \\
      & \textbf{lstm(+作文)} & 0.4189  & 0.4319  & 0.4319  & 0.6191  & 0.6191  & 0.0000  & 0.1273  & 0.2182  & 0.0009  \\
      & \textbf{bert\_CLS} & 0.4526  & 0.4579  & 0.4579  & 0.4579  & 0.4579  & 0.1818  & 0.2000  & 0.1818  & 0.1146  \\
      & \textbf{bert\_Last1avg} & 0.5500  & 0.6409  & 0.6560  & 0.6863  & 0.6863  & 0.4545  & 0.2545  & 0.2545  & 0.1250  \\
      & \textbf{bert\_CLS(+作文)} & 0.4944  & 0.4998  & 0.4998  & 0.5105  & 0.5319  & 0.3636  & 0.2545  & 0.2364  & 0.0420  \\
      & \textbf{bert\_Last1avg(+作文)} & 0.4739  & 0.4793  & 0.4793  & 0.5256  & 0.5310  & 0.4545  & 0.2727  & 0.2273  & 0.0324  \\
      \hline
    \end{tabular}%
    \begin{tablenotes}    %这行要添加, 从这开始
      \footnotesize               %这行要添加
      \item[1] R@all 表示全部小类中离题的召回
      \item[2] +作文 表示在作文数据上进行微调(原始模型使用新闻数据预训练) 
    \end{tablenotes} 
  \caption{聚类方案一指标更新}
  \label{tab:addlabel}%
\end{table}%


% Table generated by Excel2LaTeX from sheet '整理(余弦相似度,使用占比划分大小类)'
\begin{table}[htbp]\small
  \centering
    \begin{tabular}{c|c|c|c|c|c|c|c|c|c}
      \hline
      \textcolor[rgb]{ 1,  0,  0}{} & \multicolumn{1}{c}{\textbf{R@10}} & \multicolumn{1}{c}{\textbf{R@15}} & \multicolumn{1}{c}{\textbf{R@20}} & \multicolumn{1}{c}{\textbf{R@50}} & \multicolumn{1}{c}{\textbf{R@all}} & \multicolumn{1}{c}{\textbf{P@1}} & \multicolumn{1}{c}{\textbf{P@5}} & \multicolumn{1}{c}{\textbf{P@10}} & \multicolumn{1}{c}{\textbf{spearman}} \\
      \hline
      \textbf{linkage='average'} & 0.5334  & 0.6243  & 0.6296  & 0.6888  & 0.7459  & 0.2727  & 0.2727  & 0.2727  & 0.1355  \\
      \hline
      \textbf{linkage='complete'} & \textcolor[rgb]{ 1,  0,  0}{\textbf{0.6849 }} & 0.6849  & 0.6902  & 0.7312  & 0.7883  & 0.2727  & 0.3091  & 0.3000  & 0.1504  \\
      \hline
      \textbf{linkage='single'} & 0.5001  & 0.6061  & 0.6061  & 0.6781  & 0.7459  & 0.4545  & 0.2727  & 0.2545  & 0.1573  \\
      \hline
    \end{tabular}%
    \begin{tablenotes}
      \item[1] linkage='average': 取平均;linkage='complete': 取最大值;linkage='single': 取最小值;
    \end{tablenotes}
    \caption{linkage取不同参数时指标对比(doc2vec)}
  \label{tab:addlabel}%
\end{table}%


% Table generated by Excel2LaTeX from sheet '整理(余弦相似度,使用占比划分大小类)'
\begin{table}[htbp]\small
  \centering
  \begin{tabular}{c|c|c|c|c|c|c|c|c|c}
    \hline
    \multicolumn{2}{c|}{} & \textbf{R@10} & \textbf{R@15} & \textbf{R@20} & \textbf{R@50} & \textbf{P@1} & \textbf{P@5} & \textbf{P@10} & \textbf{spearman} \\
    \hline
    \multicolumn{2}{c|}{\textbf{baseline}} & 0.3917  & 0.4631  & 0.5594  & 0.7148  & 0.4545  & 0.2909  & 0.2182  & 0.1429  \\
    \hline
    \multicolumn{2}{c|}{\textbf{tfidf}} & 0.4350  & 0.4533  & 0.5041  & 0.7558  & 0.4545  & 0.3091  & 0.2364  & 0.2437  \\
    \hline
    \multicolumn{2}{c|}{\textbf{doc2vec}} & 0.3865  & 0.4381  & 0.5291  & 0.7041  & 0.5455  & 0.2727  & 0.2182  & 0.1865  \\
    \hline
    \multirow{4}[0]{*}{\textbf{分类模型}} & \textbf{lstm} & 0.3728  & 0.4062  & 0.4479  & 0.7398  & 0.4545  & 0.2727  & 0.1909  & 0.1535  \\
    & \textbf{bert\_CLS} & 0.3029  & 0.4480  & 0.5419  & 0.7344  & 0.2727  & 0.2182  & 0.1818  & 0.0960  \\
    & \textbf{bert\_Last1avg} & 0.3635  & 0.4480  & 0.4934  & 0.7291  & 0.3636  & 0.2364  & 0.2000  & 0.1101  \\
    & \textbf{bert\_Last2avg} & 0.3582  & 0.4176  & 0.4631  & 0.7291  & 0.2727  & 0.2545  & 0.1909  & 0.1189  \\
    \hline
    \multirow{6}[0]{*}{\textbf{生成模型}} & \textbf{lstm} & 0.1826  & 0.2850  & 0.4062  & 0.6684  & 0.0909  & 0.1273  & 0.1091  & 0.0156  \\
    & \textbf{lstm(+作文)} & 0.1263  & 0.2463  & 0.3676  & 0.7041  & 0.0000  & 0.0727  & 0.0818  & -0.0039  \\
    & \textbf{bert\_CLS} & 0.3476  & 0.6214  & 0.6820  & 0.8503  & 0.1818  & 0.2182  & 0.1818  & 0.1516  \\
    & \textbf{bert\_Last1avg} & 0.3393  & 0.5328  & 0.6873  & 0.8449  & 0.4545  & 0.2182  & 0.1727  & 0.1026  \\
    & \textbf{bert\_CLS(+作文)} & \textcolor[rgb]{ 1,  0,  0}{\textbf{0.5259 }} & 0.5389  & 0.5389  & 0.7504  & 0.3636  & 0.2364  & 0.2636  & 0.1266  \\
    & \textbf{bert\_Last1avg(+作文)} & 0.5107  & 0.5389  & 0.5389  & 0.7451  & 0.3636  & 0.2909  & 0.2545  & 0.1280  \\
    \hline
    \end{tabular}%
    \caption{聚类方案二(one-class)指标更新}
  \label{tab:addlabel}%
\end{table}%


% Table generated by Excel2LaTeX from sheet '整理(余弦相似度,使用占比划分大小类)'
\begin{table}[htbp]\small
  \centering
    \begin{tabular}{c|c|c|c|c|c|c|c|c|c|c}
      \hline
      \multicolumn{2}{c|}{} & \textbf{R@10} & \textbf{R@15} & \textbf{R@20} & \textbf{R@50} & \textbf{R@all} & \textbf{P@1} & \textbf{P@5} & \textbf{P@10} & \textbf{spearman} \\
      \hline
      \multicolumn{2}{c|}{\textbf{baseline}} & 0.1070  & 0.1070  & 0.1123  & 0.3351  & 0.3779  & 0.0000  & 0.0909  & 0.0545  & 0.0197  \\
      \hline
      \multicolumn{2}{c|}{\textbf{tfidf}} & \textcolor[rgb]{ .502,  .502,  .502}{0.2323 } & \textcolor[rgb]{ .502,  .502,  .502}{0.2626 } & \textcolor[rgb]{ .502,  .502,  .502}{0.2680 } & \textcolor[rgb]{ .502,  .502,  .502}{0.3339 } & \textcolor[rgb]{ .502,  .502,  .502}{0.3500 } & \textcolor[rgb]{ .502,  .502,  .502}{0.3636 } & \textcolor[rgb]{ .502,  .502,  .502}{0.2182 } & \textcolor[rgb]{ .502,  .502,  .502}{0.1273 } & \textcolor[rgb]{ .502,  .502,  .502}{0.0547 } \\
      \hline
      \multicolumn{2}{c|}{\textbf{doc2vec}} & 0.1468  & 0.2256  & 0.2256  & 0.5129  & 0.5343  & 0.0909  & 0.1273  & 0.0909  & 0.1160  \\
      \hline
      \multirow{4}[0]{*}{\textbf{分类模型}} & \textbf{lstm} & 0.0645  & 0.1184  & 0.1668  & 0.2791  & 0.3219  & 0.0909  & 0.0182  & 0.0455  & 0.0089  \\
      & \textbf{bert\_CLS} & 0.2371  & 0.2674  & 0.3280  & 0.5259  & 0.5580  & 0.1818  & 0.1636  & 0.1455  & 0.1018  \\
      & \textbf{bert\_Last1avg} & 0.1982  & 0.2285  & 0.2891  & 0.5119  & 0.5386  & 0.2727  & 0.1636  & 0.1182  & 0.0908  \\
      & \textbf{bert\_Last2avg} & 0.2231  & 0.2891  & 0.3194  & 0.5172  & 0.5333  & 0.1818  & 0.1636  & 0.1182  & 0.0960  \\
      \hline
      \multirow{6}[0]{*}{\textbf{生成模型}} & \textbf{lstm} & 0.1273  & 0.2182  & 0.2182  & 0.2699  & 0.2752  & 0.0909  & 0.0545  & 0.0636  & 0.0045  \\
      & \textbf{lstm\_微调} & 0.0766  & 0.1426  & 0.1715  & 0.2231  & 0.2392  & 0.0000  & 0.0545  & 0.0455  & 0.0009  \\
      & \textbf{bert\_CLS} & 0.3501  & 0.3554  & 0.3661  & 0.3715  & 0.3715  & 0.1818  & 0.1636  & 0.1182  & 0.1271  \\
      & \textbf{bert\_Last1avg} & 0.2549  & 0.3390  & 0.4602  & 0.6224  & 0.6331  & 0.2727  & 0.2182  & 0.1273  & 0.0413  \\
      & \textbf{bert\_CLS\_微调} & \textcolor{red}{0.3873}  & 0.4480  & 0.4533  & 0.4943  & 0.5103  & 0.2727  & 0.2545  & 0.1909  & 0.0662  \\
      & \textbf{bert\_Last1avg\_微调} & 0.3009  & 0.3062  & 0.3971  & 0.4078  & 0.4078  & 0.3636  & 0.2000  & 0.1455  & 0.0407  \\
      \hline
    \end{tabular}%
    \begin{tablenotes}    %这行要添加, 从这开始
      \footnotesize               %这行要添加
      \item[1] R@all 表示全部小类中离题的召回
    \end{tablenotes} 
    \caption{聚类方案三(prompt-independent)指标更新}
  \label{tab:addlabel}%
\end{table}%


% Table generated by Excel2LaTeX from sheet '整理(余弦相似度,使用占比划分大小类)'
\begin{table}[htbp]\small
  \centering
    \begin{tabular}{c|c|c|c|c|c|c|c|c|c}
      \hline
      \multicolumn{2}{c|}{} & \textbf{R@10} & \textbf{R@15} & \textbf{R@20} & \textbf{R@50} & \textbf{P@1} & \textbf{P@5} & \textbf{P@10} & \textbf{spearman} \\
      \hline
      \multicolumn{2}{c|}{\textbf{baseline}} & -     & -     & -     & -     & -     & -     & -     & - \\
      \hline
      \multicolumn{2}{c|}{\textbf{tfidf}} & 0.3692  & 0.4654  & 0.5496  & 0.7094  & 0.4545  & 0.2909  & 0.2182  & 0.2719  \\
      \hline
      \multicolumn{2}{c|}{\textbf{doc2vec}} & 0.2430  & 0.3175  & 0.3175  & 0.8093  & 0.0909  & 0.1273  & 0.1273  & 0.1525  \\
      \hline
      \multirow{4}[0]{*}{\textbf{分类模型}} & \textbf{lstm} & 0.2894  & 0.3487  & 0.4093  & 0.6328  & 0.6364  & 0.2364  & 0.1727  & 0.1239  \\
      & \textbf{bert\_CLS} & 0.4136  & 0.4934  & 0.5267  & 0.6738  & 0.2727  & 0.2364  & 0.2000  & 0.0835  \\
      & \textbf{bert\_Last1avg} & \textcolor[rgb]{ 1,  0,  0}{\textbf{0.4781 }} & 0.5062  & 0.5820  & 0.6988  & 0.4545  & 0.2727  & 0.2364  & 0.1102  \\
      & \textbf{bert\_Last2avg} & 0.4296  & 0.5668  & 0.5722  & 0.6988  & 0.2727  & 0.2182  & 0.2182  & 0.1165  \\
      \hline
      \multirow{6}[0]{*}{\textbf{生成模型}} & \textbf{lstm} & 0.1188  & 0.2258  & 0.2410  & 0.5829  & 0.0000  & 0.0909  & 0.0727  & 0.0161  \\
      & \textbf{lstm(+作文)} & 0.1469  & 0.2312  & 0.3608  & 0.6738  & 0.0000  & 0.0727  & 0.0909  & 0.0796  \\
      & \textbf{bert\_CLS} & 0.3273  & 0.3879  & 0.4039  & 0.7398  & 0.0909  & 0.2000  & 0.1545  & 0.1629  \\
      & \textbf{bert\_Last1avg} & 0.3523  & 0.3911  & 0.4578  & 0.7201  & 0.2727  & 0.2545  & 0.1727  & 0.2072  \\
      & \textbf{bert\_CLS(+作文)} & 0.4328  & 0.4435  & 0.4889  & 0.6488  & 0.4545  & 0.3636  & 0.2364  & 0.1981  \\
      & \textbf{bert\_Last1avg(+作文)} & 0.4328  & 0.4738  & 0.4889  & 0.6435  & 0.4545  & 0.3818  & 0.2364  & 0.2241  \\
      \hline
    \end{tabular}%
    \begin{tablenotes}    %这行要添加, 从这开始
      \footnotesize               %这行要添加
      \item [*] 无法获取prompt的baseline特征表示
    \end{tablenotes} 
    \caption{聚类方案四指标更新}
  \label{tab:addlabel}%
\end{table}%

\end{CJK}
\end{document}

% include your own bib file like this:


%\begin{thebibliography}{}

%\bibitem[\protect\citename{Aho and Ullman}1972]{Aho:72}
%Alfred~V. Aho and Jeffrey~D. Ullman.
%\newblock 1972.
%\newblock {\em The Theory of Parsing, Translation and Compiling}, volume~1.
%\newblock Prentice-{Hall}, Englewood Cliffs, NJ.

%\bibitem[\protect\citename{{American Psychological Association}}1983]{APA:83}
%{American Psychological Association}.
%\newblock 1983.
%\newblock {\em Publications Manual}.
%\newblock American Psychological Association, Washington, DC.

%\bibitem[\protect\citename{{Association for Computing Machinery}}1983]{ACM:83}
%{Association for Computing Machinery}.
%\newblock 1983.
%\newblock {\em Computing Reviews}, 24(11):503--512.

%\bibitem[\protect\citename{Chandra \bgroup et al.\egroup }1981]{Chandra:81}
%Ashok~K. Chandra, Dexter~C. Kozen, and Larry~J. Stockmeyer.
%\newblock 1981.
%\newblock Alternation.
%\newblock {\em Journal of the Association for Computing Machinery},
%  28(1):114--133.

%\bibitem[\protect\citename{Gusfield}1997]{Gusfield:97}
%Dan Gusfield.
%\newblock 1997.
%\newblock {\em Algorithms on Strings, Trees and Sequences}.
%\newblock Cambridge University Press, Cambridge, UK.

%\bibitem[\protect\citename{Rasooli and Tetreault}2015]{rasooli-tetrault-2015}
%Mohammad~Sadegh Rasooli and Joel~R. Tetreault. 2015.
%\newblock {Yara parser: {A} fast and accurate dependency parser}.
%\newblock \emph{Computing Research Repository}, arXiv:1503.06733.
%\newblock Version 2.

%\bibitem[\protect\citename{Borschinger and Johnson}2011]{borsch2011}
%Benjamin Borschinger and Mark Johnson. 2011.
%\newblock A particle filter algorithm for {B}ayesian wordsegmentation.
%\newblock In \emph{Proceedings of the Australasian Language Technology Association %Workshop 2011}, pages 10--18, Canberra, Australia.

%\end{thebibliography}

