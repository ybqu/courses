% \documentclass{article}
\documentclass[UTF8]{ctexart}
\usepackage{algorithm}
\usepackage{algorithmic}
\usepackage{CJK}
\renewcommand{\algorithmicrequire}{ \textbf{Input:}} %Use Input in the format of Algorithm
\renewcommand{\algorithmicensure}{ \textbf{Output:}} %UseOutput in the format of Algorithm
\begin{document}
\begin{algorithm}[h]
    \caption{离题检测聚类方案算法}
    \label{}
    \begin{algorithmic}[1]
      \REQUIRE ~~\\ %算法的输入参数:Input
        Number of topics $N$\\
        Model $M$ obtain context vector
      \ENSURE ~~\\ %算法的输出:Output
        $Recall@10$ for each prompts \\ 
        $R$ for all prompts \\
        \FOR{each $i \in [1,N]$}
            \STATE Get the context vector $C_j(j=0,1,...,N)$ of all compositions under the prompt $P_i$ by model $M$;\
            \STATE Use AgglomerativeClustering algorithm to cluster $C_j$, get M clusters $R_k(k=0,1,...,M)$\
            \FOR{each $k \in R_k$}
              \IF{$Num(k)$ \textgreater 5}
                \STATE $Samples$.append(k)\   % 大概率不离题
              \ELSE
                \STATE $MaybeOT$.append(k)\   % 大概率离题(孤立点)
              \ENDIF
            \ENDFOR
            \FOR{each $ot \in MaybeOT$}
              \STATE Calculate the distance between $ot$ and $Samples$ separately, get $D_j(j=0,1,...,len(Samples))$\
              \STATE $D = min(D_j)$   % 当前孤立点的最小距离
            \ENDFOR
            \STATE Sort all $D$, calculate $Recall@10$, get the $Recall_i$ of $P_i$\
        \ENDFOR

        \STATE Take the average of all $Recall_i$ to get $R$\
        \STATE $R=\frac{\sum_{i=0}^{10} Recall_i}{10}$\
    \end{algorithmic}
\end{algorithm}
\end{document}