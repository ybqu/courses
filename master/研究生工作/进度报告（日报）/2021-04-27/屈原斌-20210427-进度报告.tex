%
% File coling2020.tex
%
% Contact: feiliu@cs.ucf.edu & liang.huang.sh@gmail.com
%% Based on the style files for COLING-2018, which were, in turn,
%% Based on the style files for COLING-2016, which were, in turn,
%% Based on the style files for COLING-2014, which were, in turn,
%% Based on the style files for ACL-2014, which were, in turn,
%% Based on the style files for ACL-2013, which were, in turn,
%% Based on the style files for ACL-2012, which were, in turn,
%% based on the style files for ACL-2011, which were, in turn, 
%% based on the style files for ACL-2010, which were, in turn, 
%% based on the style files for ACL-IJCNLP-2009, which were, in turn,
%% based on the style files for EACL-2009 and IJCNLP-2008...

%% Based on the style files for EACL 2006 by 
%%e.agirre@ehu.es or Sergi.Balari@uab.es
%% and that of ACL 08 by Joakim Nivre and Noah Smith

\documentclass[11pt]{article}
\usepackage{coling2020}
\usepackage{times}
\usepackage{url}
\usepackage{latexsym}
\usepackage{indentfirst}

\usepackage{times}
\usepackage{latexsym}
\usepackage{times}
\usepackage{soul}
\usepackage{url}
\usepackage{amsmath}
\usepackage{amsthm}
\usepackage{booktabs}
\usepackage{algorithm}
\usepackage{algorithmic}
\usepackage{amssymb}
\usepackage{longtable}
\usepackage{graphicx}
\usepackage{CJK}
\usepackage{multirow}
\usepackage{color}

%\setlength\titlebox{5cm}
\colingfinalcopy % Uncomment this line for the final submission

% You can expand the titlebox if you need extra space
% to show all the authors. Please do not make the titlebox
% smaller than 5cm (the original size); we will check this
% in the camera-ready version and ask you to change it back.


\title{2021年04月25日进度汇报}

\author{屈原斌 \\
  首都师范大学 \\
    {\tt ybqu@cnu.edu.cn}}

\date{}

\begin{document}
\begin{CJK}{UTF8}{gkai}

\maketitle
\CJKindent
%\begin{abstract}

%\end{abstract}

\section{今日进度}


\begin{itemize}
  \item [1.] 更新离题检测实验
\end{itemize}

\section{工作详述}
\begin{itemize}
  \item 实验数据:
  \begin{itemize}
    \item ICLE数据集,11个主题,共827篇作文,离题:切题=51:776(1:15)
  \end{itemize}
  \item 实验方案:
  \begin{itemize}
    \item 聚类方案一:
    \begin{itemize}
      \item 计算聚类结果中可能离题的类与范文类的相似度
      \item distance\_threshold从最小到最大距离进行调参,example\_threshold从1到10进行调参
      \item 实验结果见表1
      \item 结论:
      \begin{itemize}
        \item doc2vec表示时指标最优
      \end{itemize}
    \end{itemize}
    \item 聚类方案二(one-class):
    \begin{itemize}
      \item 计算全部作文与它们的质心的相似度,指标见表2
      \item 结论:
      \begin{itemize}
        \item bert分类模型指标最优
      \end{itemize}
    \end{itemize}
    \item 聚类方案三(Prompt-independent)
    \begin{itemize}
      \item 五折交叉验证,数据划分见表3
      \item 指标见表4
      \item 结论:
      \begin{itemize}
        \item bert生成模型指标最优
        \item [!] 开发集调参后测试集部分主题聚类后没有小类
      \end{itemize}
    \end{itemize}
    
  \end{itemize}
\end{itemize}

% Table generated by Excel2LaTeX from sheet '整理'
\begin{table}[htbp]
  \centering
  \begin{tabular}{c|c|c|c|c|c|c|c|c|c}
    \hline
    \multicolumn{2}{c|}{} & \multicolumn{1}{c}{\textbf{R@10}} & \multicolumn{1}{c}{\textbf{R@15}} & \multicolumn{1}{c|}{\textbf{R@20}} & \multicolumn{1}{c}{\textbf{R@50}} & \multicolumn{1}{c}{\textbf{P@1}} & \multicolumn{1}{c}{\textbf{P@5}} & \multicolumn{1}{c}{\textbf{P@10}} & \multicolumn{1}{c}{\textbf{spearman}} \\
    \hline
    \multicolumn{2}{c|}{\textbf{baseline}} & 0.5334  & 0.5464  & 0.6676  & 0.7321  & 0.2727  & 0.3091  & 0.2727  & 0.1355  \\
    \hline
    \multicolumn{2}{c|}{\textbf{tfidf}} & 0.4350  & 0.4836  & 0.5291  & 0.7861  & 0.4545  & 0.3273  & 0.2364  & 0.2260  \\
    \hline
    \multicolumn{2}{c|}{\textbf{doc2vec}} & \textcolor[rgb]{ 1,  0,  0}{\textbf{0.6646 }} & 0.6646  & 0.6797  & 0.7464  & 0.1818  & 0.2364  & 0.2909  & 0.1698  \\
    \hline
    \multirow{2}[0]{*}{\textbf{分类模型}} & \textbf{lstm} & 0.4150  & 0.4204  & 0.5173  & 0.6318  & 0.3636  & 0.2000  & 0.2000  & 0.1257  \\
    & \textbf{bert} & 0.6146  & 0.6146  & 0.6449  & 0.7661  & 0.2727  & 0.2545  & 0.2818  & 0.1790  \\
    \hline
    \multirow{2}[0]{*}{\textbf{生成模型}} & \textbf{lstm} & 0.5348  & 0.5478  & 0.5629  & 0.7562  & 0.0909  & 0.2000  & 0.2273  & 0.0137  \\
    & \textbf{bert} & 0.5355  & 0.5485  & 0.5485  & 0.6576  & 0.2727  & 0.2364  & 0.2636  & 0.0657  \\
    \hline
  \end{tabular}%
  \caption{聚类方案一指标更新}
  \label{tab:addlabel}%
\end{table}%


% Table generated by Excel2LaTeX from sheet '整理'
% Table generated by Excel2LaTeX from sheet '整理'
\begin{table}[htbp]
  \centering
  \begin{tabular}{c|c|c|c|c|c|c|c|c|c}
    \hline
    \multicolumn{2}{c}{} & \textbf{R@10} & \textbf{R@15} & \textbf{R@20} & \textbf{R@50} & \textbf{P@1} & \textbf{P@5} & \textbf{P@10} & \textbf{spearman} \\
    \hline
    \multicolumn{2}{c}{\textbf{baseline}} & 0.3917  & 0.4631  & 0.5594  & 0.7148  & 0.4545  & 0.2909  & 0.2182  & 0.1429  \\
    \hline
    \multicolumn{2}{c}{\textbf{tfidf}} & 0.4350  & 0.4533  & 0.5041  & 0.7558  & 0.4545  & 0.3091  & 0.2364  & 0.2437  \\
    \hline
    \multicolumn{2}{c}{\textbf{doc2vec}} & 0.3865  & 0.4381  & 0.5291  & 0.7041  & 0.5455  & 0.2727  & 0.2182  & 0.1865  \\
    \hline
    \multirow{2}[0]{*}{\textbf{分类模型}} & \textbf{lstm} & 0.3728  & 0.4062  & 0.4479  & 0.7398  & 0.4545  & 0.2727  & 0.1909  & 0.1535  \\
    & \textbf{bert} & 0.3029  & 0.4480  & 0.5419  & 0.7344  & 0.2727  & 0.2182  & 0.1818  & 0.0960  \\
    \hline
    \multirow{2}[0]{*}{\textbf{生成模型}} & \textbf{lstm} & 0.1263  & 0.2463  & 0.3676  & 0.7041  & 0.0000  & 0.0727  & 0.0818  & \textcolor[rgb]{ 0,  .69,  .314}{\textbf{-0.0039 }} \\
    & \textbf{bert} & \textcolor[rgb]{ 1,  0,  0}{\textbf{0.5259 }} & 0.5389  & 0.5389  & 0.7504  & 0.3636  & 0.2364  & 0.2636  & 0.1266  \\
    \hline
  \end{tabular}%
  \caption{聚类方案二(one-class)指标更新}
  \label{tab:addlabel}%
\end{table}%



% Table generated by Excel2LaTeX from sheet '整理'
\begin{table}[htbp]
  \centering
  \begin{tabular}{c|c|c}
    \hline
    & 主题数   & 作文数 \\
    \hline
    一折    & 1     & 237 \\
    \hline
    二折    & 2     & 144 \\
    \hline
    三折    & 2     & 151 \\
    \hline
    四折    & 3     & 141 \\
    \hline
    五折    & 3     & 154 \\
    \hline
  \end{tabular}%
  \caption{五折划分数据}
  \label{tab:addlabel}%
\end{table}%


% Table generated by Excel2LaTeX from sheet '整理'
\begin{table}[htbp]
  \centering
  \begin{tabular}{c|c|c|c|c|c|c|c|c|c}
    \hline
    \multicolumn{2}{c|}{} & \textbf{R@10} & \textbf{R@15} & \textbf{R@20} & \textbf{R@50} & \textbf{P@1} & \textbf{P@5} & \textbf{P@10} & \textbf{spearman} \\
    \hline
    \multicolumn{2}{c|}{\textbf{baseline}} & 0.1524  & 0.1676  & 0.1676  & 0.3155  & 0.1818  & 0.1273  & 0.0909  & 0.0855  \\
    \hline
    \multicolumn{2}{c|}{\textbf{tfidf}} & 0.3138  & 0.3624  & 0.3775  & 0.5437  & 0.3636  & 0.2727  & 0.2000  & 0.1711  \\
    \hline
    \multicolumn{2}{c|}{\textbf{doc2vec}} & 0.2098  & 0.2205  & 0.2259  & 0.3328  & 0.0909  & 0.1273  & 0.1091  & 0.0517  \\
    \hline
    \multirow{2}[0]{*}{\textbf{分类模型}} & \textbf{lstm} & 0.1023  & 0.1380  & 0.1683  & 0.2645  & 0.0909  & 0.0364  & 0.0455  & \textcolor[rgb]{ 0,  .69,  .314}{\textbf{-0.0061 }} \\
    & \textbf{bert} & 0.1679  & 0.3497  & 0.3800  & 0.4566  & 0.1818  & 0.1818  & 0.1091  & 0.1551  \\
    \hline
    \multirow{2}[0]{*}{\textbf{生成模型}} & \textbf{lstm} & 0.2194  & 0.2816  & 0.3270  & 0.6067  & 0.0000  & 0.0545  & 0.1091  & \textcolor[rgb]{ 0,  .69,  .314}{\textbf{-0.0386 }} \\
    & \textbf{bert} & \textcolor[rgb]{ 1,  0,  0}{\textbf{0.4241 }} & 0.4847  & 0.5052  & 0.6630  & 0.3636  & 0.3091  & 0.2182  & 0.1149  \\
    \hline
  \end{tabular}%
  \caption{Prompt-independent方案指标更新}
  \label{tab:addlabel}%
\end{table}%


\end{CJK}
\end{document}

% include your own bib file like this:


%\begin{thebibliography}{}

%\bibitem[\protect\citename{Aho and Ullman}1972]{Aho:72}
%Alfred~V. Aho and Jeffrey~D. Ullman.
%\newblock 1972.
%\newblock {\em The Theory of Parsing, Translation and Compiling}, volume~1.
%\newblock Prentice-{Hall}, Englewood Cliffs, NJ.

%\bibitem[\protect\citename{{American Psychological Association}}1983]{APA:83}
%{American Psychological Association}.
%\newblock 1983.
%\newblock {\em Publications Manual}.
%\newblock American Psychological Association, Washington, DC.

%\bibitem[\protect\citename{{Association for Computing Machinery}}1983]{ACM:83}
%{Association for Computing Machinery}.
%\newblock 1983.
%\newblock {\em Computing Reviews}, 24(11):503--512.

%\bibitem[\protect\citename{Chandra \bgroup et al.\egroup }1981]{Chandra:81}
%Ashok~K. Chandra, Dexter~C. Kozen, and Larry~J. Stockmeyer.
%\newblock 1981.
%\newblock Alternation.
%\newblock {\em Journal of the Association for Computing Machinery},
%  28(1):114--133.

%\bibitem[\protect\citename{Gusfield}1997]{Gusfield:97}
%Dan Gusfield.
%\newblock 1997.
%\newblock {\em Algorithms on Strings, Trees and Sequences}.
%\newblock Cambridge University Press, Cambridge, UK.

%\bibitem[\protect\citename{Rasooli and Tetreault}2015]{rasooli-tetrault-2015}
%Mohammad~Sadegh Rasooli and Joel~R. Tetreault. 2015.
%\newblock {Yara parser: {A} fast and accurate dependency parser}.
%\newblock \emph{Computing Research Repository}, arXiv:1503.06733.
%\newblock Version 2.

%\bibitem[\protect\citename{Borschinger and Johnson}2011]{borsch2011}
%Benjamin Borschinger and Mark Johnson. 2011.
%\newblock A particle filter algorithm for {B}ayesian wordsegmentation.
%\newblock In \emph{Proceedings of the Australasian Language Technology Association %Workshop 2011}, pages 10--18, Canberra, Australia.

%\end{thebibliography}

