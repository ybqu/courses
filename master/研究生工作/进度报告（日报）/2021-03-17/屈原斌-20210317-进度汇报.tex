%
% File coling2020.tex
%
% Contact: feiliu@cs.ucf.edu & liang.huang.sh@gmail.com
%% Based on the style files for COLING-2018, which were, in turn,
%% Based on the style files for COLING-2016, which were, in turn,
%% Based on the style files for COLING-2014, which were, in turn,
%% Based on the style files for ACL-2014, which were, in turn,
%% Based on the style files for ACL-2013, which were, in turn,
%% Based on the style files for ACL-2012, which were, in turn,
%% based on the style files for ACL-2011, which were, in turn, 
%% based on the style files for ACL-2010, which were, in turn, 
%% based on the style files for ACL-IJCNLP-2009, which were, in turn,
%% based on the style files for EACL-2009 and IJCNLP-2008...

%% Based on the style files for EACL 2006 by 
%%e.agirre@ehu.es or Sergi.Balari@uab.es
%% and that of ACL 08 by Joakim Nivre and Noah Smith

\documentclass[11pt]{article}
\usepackage{coling2020}
\usepackage{times}
\usepackage{url}
\usepackage{latexsym}
\usepackage{indentfirst}

\usepackage{times}
\usepackage{latexsym}
\usepackage{times}
\usepackage{soul}
\usepackage{url}
\usepackage{amsmath}
\usepackage{amsthm}
\usepackage{booktabs}
\usepackage{algorithm}
\usepackage{algorithmic}
\usepackage{amssymb}
\usepackage{longtable}
\usepackage{graphicx}
\usepackage{CJK}
\usepackage{multirow}
\usepackage{color}

%\setlength\titlebox{5cm}
\colingfinalcopy % Uncomment this line for the final submission

% You can expand the titlebox if you need extra space
% to show all the authors. Please do not make the titlebox
% smaller than 5cm (the original size); we will check this
% in the camera-ready version and ask you to change it back.


\title{2021年03月17日进度汇报}

\author{屈原斌 \\
  首都师范大学 \\
    {\tt ybqu@cnu.edu.cn}}

\date{}

\begin{document}
\begin{CJK}{UTF8}{gkai}

\maketitle
\CJKindent
%\begin{abstract}

%\end{abstract}

\section{今日进度}


\begin{itemize}
\item [1.] 根据篇章表示做评分
\item [2.] 增加中文数据集重新训练生成模型
\end{itemize}

\section{工作详述}
\subsection{评分结果}

\begin{itemize}
  \item 评价指标:
  \begin{itemize}
    \item $S_1 = \frac{1}{N} \sum_{A_{j} \neq E_{j}^{\prime}} 1$
    \item $S_2 = \frac{1}{N} \sum_{i=1}^{N} \left| A_j-E_j\right|$
    \item $S_3 = \frac{1}{N} \sum_{i=1}^{N} \left(A_j-E_j\right)^2$
    \item PC: Pearson’s correlation
  \end{itemize}
  \item 指标见表1:
  \item 结论:
  \begin{itemize}
    \item 使用Bert Whitening后效果变差
    \item doc2vec表示$S_1$、$S_2$指标最优
  \end{itemize}
\end{itemize}

% Table generated by Excel2LaTeX from sheet 'Sheet3'
\begin{table}[htbp]\small
  \centering
  \begin{tabular}{c|c|cccc}
    \hline
    \multicolumn{2}{c|}{} & \textbf{S1} & \textbf{S2} & \textbf{S3} & \textbf{PC} \\
    \hline
    & \textbf{doc2vec} & \textcolor[rgb]{ 1,  0,  0}{\textbf{0.5590 }} & \textcolor[rgb]{ 1,  0,  0}{\textbf{0.3476 }} & 0.2563  & 0.2047  \\
    \multirow{3}[0]{*}{分类模型} & \textbf{habilstm} & 0.7060  & 0.5590  & 0.5771  & 0.0821  \\
    & \textbf{bert\_cls} & 0.6289  & 0.3627  & 0.2373  & \textcolor[rgb]{ 1,  0,  0}{\textbf{0.2108 }} \\
    & \textbf{bert\_whitening} & 0.7831  & 0.5259  & 0.3979  & 0.0950  \\
    \hline
    \multirow{3}[0]{*}{生成模型} & \textbf{lstmabs} & 0.6867  & 0.3807  & \textcolor[rgb]{ 1,  0,  0}{\textbf{0.2331 }} & 0.1295  \\
    & \textbf{bertabs\_cls} & 0.5723  & 0.3536  & 0.2575  & 0.1361  \\
    & \textbf{bertabs\_whitening} & 0.8181  & 0.6566  & 0.6241  & 0.0743  \\
    \hline
  \end{tabular}%
  \caption{Add caption}
  \label{tab:addlabel}%
\end{table}%



% Table generated by Excel2LaTeX from sheet '中文数据集'
% \begin{figure*}[htbp]\small
%   \centering
%   \includegraphics[width=1.0\linewidth]{zb.png}
%   \caption{生成样例2}
%   \label{framework}
% \end{figure*}

%\bibliography{reference}
%\bibliographystyle{coling}
%\bibliography{coling2020}

\end{CJK}
\end{document}


% include your own bib file like this:


%\begin{thebibliography}{}

%\bibitem[\protect\citename{Aho and Ullman}1972]{Aho:72}
%Alfred~V. Aho and Jeffrey~D. Ullman.
%\newblock 1972.
%\newblock {\em The Theory of Parsing, Translation and Compiling}, volume~1.
%\newblock Prentice-{Hall}, Englewood Cliffs, NJ.

%\bibitem[\protect\citename{{American Psychological Association}}1983]{APA:83}
%{American Psychological Association}.
%\newblock 1983.
%\newblock {\em Publications Manual}.
%\newblock American Psychological Association, Washington, DC.

%\bibitem[\protect\citename{{Association for Computing Machinery}}1983]{ACM:83}
%{Association for Computing Machinery}.
%\newblock 1983.
%\newblock {\em Computing Reviews}, 24(11):503--512.

%\bibitem[\protect\citename{Chandra \bgroup et al.\egroup }1981]{Chandra:81}
%Ashok~K. Chandra, Dexter~C. Kozen, and Larry~J. Stockmeyer.
%\newblock 1981.
%\newblock Alternation.
%\newblock {\em Journal of the Association for Computing Machinery},
%  28(1):114--133.

%\bibitem[\protect\citename{Gusfield}1997]{Gusfield:97}
%Dan Gusfield.
%\newblock 1997.
%\newblock {\em Algorithms on Strings, Trees and Sequences}.
%\newblock Cambridge University Press, Cambridge, UK.

%\bibitem[\protect\citename{Rasooli and Tetreault}2015]{rasooli-tetrault-2015}
%Mohammad~Sadegh Rasooli and Joel~R. Tetreault. 2015.
%\newblock {Yara parser: {A} fast and accurate dependency parser}.
%\newblock \emph{Computing Research Repository}, arXiv:1503.06733.
%\newblock Version 2.

%\bibitem[\protect\citename{Borschinger and Johnson}2011]{borsch2011}
%Benjamin Borschinger and Mark Johnson. 2011.
%\newblock A particle filter algorithm for {B}ayesian wordsegmentation.
%\newblock In \emph{Proceedings of the Australasian Language Technology Association %Workshop 2011}, pages 10--18, Canberra, Australia.

%\end{thebibliography}

