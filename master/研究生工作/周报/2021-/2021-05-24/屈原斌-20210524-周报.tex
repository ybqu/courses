%
% File coling2020.tex
%
% Contact: feiliu@cs.ucf.edu & liang.huang.sh@gmail.com
%% Based on the style files for COLING-2018, which were, in turn,
%% Based on the style files for COLING-2016, which were, in turn,
%% Based on the style files for COLING-2014, which were, in turn,
%% Based on the style files for ACL-2014, which were, in turn,
%% Based on the style files for ACL-2013, which were, in turn,
%% Based on the style files for ACL-2012, which were, in turn,
%% based on the style files for ACL-2011, which were, in turn, 
%% based on the style files for ACL-2010, which were, in turn, 
%% based on the style files for ACL-IJCNLP-2009, which were, in turn,
%% based on the style files for EACL-2009 and IJCNLP-2008...

%% Based on the style files for EACL 2006 by 
%%e.agirre@ehu.es or Sergi.Balari@uab.es
%% and that of ACL 08 by Joakim Nivre and Noah Smith

\documentclass[11pt]{article}
\usepackage{coling2020}
\usepackage{times}
\usepackage{url}
\usepackage{latexsym}
\usepackage{indentfirst}

\usepackage{times}
\usepackage{latexsym}
\usepackage{times}
\usepackage{soul}
\usepackage{url}
\usepackage{amsmath}
\usepackage{amsthm}
\usepackage{booktabs}
\usepackage{algorithm}
\usepackage{algorithmic}
\usepackage{amssymb}
\usepackage{longtable}
\usepackage{graphicx}
\usepackage{CJK}
\usepackage{multirow}
\usepackage{color}

%\setlength\titlebox{5cm}
\colingfinalcopy % Uncomment this line for the final submission

% You can expand the titlebox if you need extra space
% to show all the authors. Please do not make the titlebox
% smaller than 5cm (the original size); we will check this
% in the camera-ready version and ask you to change it back.


\title{周报,2021年05月24日}

\author{屈原斌 \\
  首都师范大学 \\
    {\tt ybqu@cnu.edu.cn}}

\date{}

\begin{document}
\begin{CJK}{UTF8}{gkai}

\maketitle
\CJKindent
%\begin{abstract}

%\end{abstract}
\section{上周计划}


\begin{itemize}
  \item [1.] 通用场景离题检测实验更新
  \item [2.] 标题-正文匹配实验更新
  \item [2.] 成语古诗文检错实验更新
\end{itemize}


\section{上周计划执行情况}
\begin{itemize}
\item [1.] [$\times$] 中文实验未更新 
\item [2.] [$\checkmark$] 
\item [3.] [$\checkmark$] 
\end{itemize}

\section{本周部分重点工作详述}

\subsection{通用场景离题检测实验}

\begin{itemize}
  \item 数据集:
  \begin{itemize}
    \item ICLE数据集,共10个主题,离题:不离题=51:733(1:14)
  \end{itemize}
  \item 实验策略:
  \begin{itemize}
    \item 计算测试集中任意两篇作文之间的相似度
    \item 对每篇作文,选择相似度最大的k个值取平均,作为当前作文评分
    \item 对所有作文,按照评分从小到大排序
  \end{itemize}
  \item 指标见表1
  \item 结论:
  \begin{itemize}
    \item 
  \end{itemize}
\end{itemize}

% Table generated by Excel2LaTeX from sheet 'baseline_,'
\begin{table}[htbp]
  \centering
  
  \resizebox*{\textwidth}{!}{
    \begin{tabular}{c|c|cccccccccc}
      \hline
      &       & \multicolumn{1}{l}{\textbf{R@10}} & \multicolumn{1}{l}{\textbf{R@20}} & \multicolumn{1}{l}{\textbf{R@50}} & \multicolumn{1}{l}{\textbf{P@1}} & \multicolumn{1}{l}{\textbf{P@5}} & \multicolumn{1}{l}{\textbf{P@10}} & \multicolumn{1}{l}{\textbf{spearman}} & \multicolumn{1}{l}{\textbf{spearman@10}} & \multicolumn{1}{l}{\textbf{ndcg}} & \multicolumn{1}{l}{\textbf{ndcg@10}} \\
      \hline
      \multirow{2}[0]{*}{\textbf{TFIDF}} & \textbf{k=5} & 0.4392 & 0.6153 & 0.8314 & 0.5   & 0.32  & 0.24  & 0.2573 & 0.3817 & 0.9775 & 0.8244 \\
      & \textbf{k=10} & 0.4785 & 0.6153 & 0.8314 & 0.5   & 0.32  & 0.26  & 0.2635 & 0.2963 & 0.978 & 0.8196 \\
      \hline
      \multirow{2}[0]{*}{\textbf{Skip-gram}} & \textbf{k=5} & 0.5002 & 0.5902 & 0.7412 & 0.3   & 0.3   & 0.26  & 0.2143 & 0.166 & 0.9786 & 0.8248 \\
      & \textbf{k=10} & 0.5002 & 0.6294 & 0.7412 & 0.3   & 0.32  & 0.26  & 0.2162 & 0.1327 & 0.9764 & 0.8231 \\
      \hline
      \multirow{2}[0]{*}{\textbf{Doc2vec}} & \textbf{k=5} & 0.3791 & 0.6094 & 0.7471 & 0.5   & 0.28  & 0.2   & 0.1913 & 0.2551 & 0.9734 & 0.8511 \\
      & \textbf{k=10} & 0.3791 & 0.5761 & 0.7471 & 0.5   & 0.26  & 0.2   & 0.2007 & 0.2213 & 0.9746 & 0.8423 \\
      \hline
      \multirow{2}[0]{*}{\textbf{Feature-vec}} & \textbf{k=5} & 0.4308 & 0.6486 & 0.7863 & 0.4   & 0.3   & 0.24  & 0.1873 & 0.248 & 0.9753 & 0.8217 \\
      & \textbf{k=10} & 0.4308 & 0.6486 & 0.7863 & 0.5   & 0.32  & 0.24  & 0.2112 & 0.2883 & 0.9771 & 0.8155 \\
      \hline
      \multirow{2}[0]{*}{\textbf{Bert-Classification}} & \textbf{k=5} & 0.4059 & 0.6427 & 0.9529 & 0.2   & 0.22  & 0.23  & 0.2601 & 0.0341 & 0.9821 & 0.8434 \\
      & \textbf{k=10} & 0.4059 & 0.5702 & 0.9137 & 0.2   & 0.22  & 0.23  & 0.2496 & 0.0518 & 0.9818 & 0.8412 \\
      \hline
      \multirow{2}[0]{*}{\textbf{Bert-Gen}} & \textbf{k=5} & 0.3857 & 0.5702 & 0.7627 & 0.2   & 0.3   & 0.21  & 0.1728 & 0.0513 & 0.9723 & 0.863 \\
      & \textbf{k=10} & 0.4191 & 0.5643 & 0.7627 & 0.2   & 0.3   & 0.22  & 0.1706 & 0.0061 & 0.9736 & 0.8624 \\
      \hline
    \end{tabular}}%
    \caption{离题实验更新结果}
  \label{tab:addlabel}%
\end{table}%


\subsection{标题-正文匹配实验}
\begin{itemize}
  \item 数据集:
  \begin{itemize}
    \item 共341篇测试作文,符合题意:不符题意=47:294(1:6)
  \end{itemize}
  \item 实验策略:
  \begin{itemize}
    \item 计算当前作文与其他作文之间的相似度,取topk的作文标题
    \item 使用doc2vec分别获取当前作文标题表示$h$与topk作文标题表示$h_n, n=1,2,...,k$,对topk的作文标题表示取平均$h_{avg}$
    \item 计算$h$与$h_{avg}$的相似度$S$
    \item 根据阈值判断是否符合题意
  \end{itemize}
  \item 指标见表2
  \item 结论:
  \begin{itemize}
    \item 使用Last1avg表示优于CLS
    \item k值增大指标上升
  \end{itemize}
\end{itemize}

% Table generated by Excel2LaTeX from sheet '标题-正文匹配'
\begin{table}[htbp]\small
  \centering
  \begin{tabular}{c|c|c|c|ccc|ccc}
    \hline
    \multicolumn{2}{c}{} &       &       & \multicolumn{3}{c}{\textbf{符合题意}} & \multicolumn{3}{c}{\textbf{不符题意}} \\
    \hline
    \multicolumn{2}{c}{} & \textbf{最优阈值} & \textbf{Accuracy} & \textbf{precision} & \textbf{recall} & \textbf{F1-score} & \textbf{precision} & \textbf{recall} & \textbf{F1-score} \\
    \hline
    \multirow{2}[0]{*}{\textbf{BERT-Gen\_CLS}} & \textbf{k=5} & 0.50  & 0.7214  & 0.9028  & 0.7585  & 0.8244  & 0.2447  & 0.4894  & 0.3262  \\
    & \textbf{k=10} & 0.60  & 0.6540  & 0.9190  & 0.6565  & 0.7659  & 0.2290  & 0.6383  & 0.3371  \\
    \hline
    \multirow{2}[0]{*}{\textbf{BERT-Gen\_Last1avg\_}} & \textbf{k=5} & 0.56  & 0.6686  & 0.9132  & 0.6803  & 0.7797  & 0.2295  & 0.5957  & \textcolor[rgb]{ 1,  0,  0}{\textbf{0.3314 }} \\
    & \textbf{k=10} & 0.60  & 0.6452  & 0.9220  & 0.6429  & 0.7575  & 0.2279  & 0.6596  & 0.3388  \\
    \hline
  \end{tabular}%
  \caption{标题-正文匹配实验结果}
  \label{tab:addlabel}%
\end{table}%


\section{下周计划}
\begin{itemize}
\item [1.] [*****] TFIDF/分布式表示验证实验
\item [2.] [***] 标题-正文匹配测试集构建,更新三档实验结果
\item [3.] [***] 补全中英文离题检测实验
\item [4.] [***] 成语古诗文检错替换方案
\end{itemize}
%\bibliography{reference}
%\bibliographystyle{coling}
%\bibliography{coling2020}

\end{CJK}
\end{document}


% include your own bib file like this:


%\begin{thebibliography}{}

%\bibitem[\protect\citename{Aho and Ullman}1972]{Aho:72}
%Alfred~V. Aho and Jeffrey~D. Ullman.
%\newblock 1972.
%\newblock {\em The Theory of Parsing, Translation and Compiling}, volume~1.
%\newblock Prentice-{Hall}, Englewood Cliffs, NJ.

%\bibitem[\protect\citename{{American Psychological Association}}1983]{APA:83}
%{American Psychological Association}.
%\newblock 1983.
%\newblock {\em Publications Manual}.
%\newblock American Psychological Association, Washington, DC.

%\bibitem[\protect\citename{{Association for Computing Machinery}}1983]{ACM:83}
%{Association for Computing Machinery}.
%\newblock 1983.
%\newblock {\em Computing Reviews}, 24(11):503--512.

%\bibitem[\protect\citename{Chandra \bgroup et al.\egroup }1981]{Chandra:81}
%Ashok~K. Chandra, Dexter~C. Kozen, and Larry~J. Stockmeyer.
%\newblock 1981.
%\newblock Alternation.
%\newblock {\em Journal of the Association for Computing Machinery},
%  28(1):114--133.

%\bibitem[\protect\citename{Gusfield}1997]{Gusfield:97}
%Dan Gusfield.
%\newblock 1997.
%\newblock {\em Algorithms on Strings, Trees and Sequences}.
%\newblock Cambridge University Press, Cambridge, UK.

%\bibitem[\protect\citename{Rasooli and Tetreault}2015]{rasooli-tetrault-2015}
%Mohammad~Sadegh Rasooli and Joel~R. Tetreault. 2015.
%\newblock {Yara parser: {A} fast and accurate dependency parser}.
%\newblock \emph{Computing Research Repository}, arXiv:1503.06733.
%\newblock Version 2.

%\bibitem[\protect\citename{Borschinger and Johnson}2011]{borsch2011}
%Benjamin Borschinger and Mark Johnson. 2011.
%\newblock A particle filter algorithm for {B}ayesian wordsegmentation.
%\newblock In \emph{Proceedings of the Australasian Language Technology Association %Workshop 2011}, pages 10--18, Canberra, Australia.

%\end{thebibliography}

