%
% File coling2020.tex
%
% Contact: feiliu@cs.ucf.edu & liang.huang.sh@gmail.com
%% Based on the style files for COLING-2018, which were, in turn,
%% Based on the style files for COLING-2016, which were, in turn,
%% Based on the style files for COLING-2014, which were, in turn,
%% Based on the style files for ACL-2014, which were, in turn,
%% Based on the style files for ACL-2013, which were, in turn,
%% Based on the style files for ACL-2012, which were, in turn,
%% based on the style files for ACL-2011, which were, in turn, 
%% based on the style files for ACL-2010, which were, in turn, 
%% based on the style files for ACL-IJCNLP-2009, which were, in turn,
%% based on the style files for EACL-2009 and IJCNLP-2008...

%% Based on the style files for EACL 2006 by 
%%e.agirre@ehu.es or Sergi.Balari@uab.es
%% and that of ACL 08 by Joakim Nivre and Noah Smith

\documentclass[11pt]{article}
\usepackage{coling2020}
\usepackage{times}
\usepackage{url}
\usepackage{latexsym}
\usepackage{indentfirst}

\usepackage{times}
\usepackage{latexsym}
\usepackage{times}
\usepackage{soul}
\usepackage{url}
\usepackage{amsmath}
\usepackage{amsthm}
\usepackage{booktabs}
\usepackage{algorithm}
\usepackage{algorithmic}
\usepackage{amssymb}
\usepackage{longtable}
\usepackage{graphicx}
\usepackage{CJK}
\usepackage{multirow}
\usepackage{color}

%\setlength\titlebox{5cm}
\colingfinalcopy % Uncomment this line for the final submission

% You can expand the titlebox if you need extra space
% to show all the authors. Please do not make the titlebox
% smaller than 5cm (the original size); we will check this
% in the camera-ready version and ask you to change it back.


\title{周报,2021年04月05日}

\author{屈原斌 \\
  首都师范大学 \\
    {\tt ybqu@cnu.edu.cn}}

\date{}

\begin{document}
\begin{CJK}{UTF8}{gkai}

\maketitle
\CJKindent
%\begin{abstract}

%\end{abstract}
\section{上周计划}


\begin{itemize}
  \item [1.] 更新baseline相关实验
\end{itemize}


\section{上周计划执行情况}
\begin{itemize}
\item [1.] [$\checkmark$] 
\end{itemize}

\section{本周部分重点工作详述}

\subsection{baseline实验}

\begin{itemize}
  \item 数据集:
  \begin{itemize}
    \item ICLE,13个主题共830篇作文,各分数段作文分布见表1
  \end{itemize}
  \item 回归方案:五折交叉验证,实验结果见表2
  \item 分类方案:五折交叉验证
  \begin{itemize}
    \item 五分类:按照2.0-4.0五个分数档进行分类,实验结果见表3
    \item 三分类:将2.0-3.5的作文划分为离题,4.0作文划分为不离题,实验结果见表4
  \end{itemize}
  \item 结论:
  \begin{itemize}
    \item 回归方案结果差于论文中的指标,大部分的预测结果都集中在3.0-4.0
    \item 使用c-SVC进行分类时离题的指标很低,大部分作文都分类到高分,2.5和3.5分都预测不到
    \item 使用nc-SVC进行二分类时离题指标可以达到0.5319
  \end{itemize}
\end{itemize}

% Table generated by Excel2LaTeX from sheet '英文数据集'
\begin{table}[htbp]
  \centering
  \begin{tabular}{c|c|c|c|c|c|c|c}
    \hline
    \textbf{score} & \textbf{1.0} & \textbf{1.5} & \textbf{2.0} & \textbf{2.5} & \textbf{3.0} & \textbf{3.5} & \textbf{4.0} \\
    \hline
    \textbf{作文数} & 0     & 0     & 8     & 44    & 105   & 230   & 443 \\
    \hline
  \end{tabular}%
  \caption{各分数档作文分布}
  \label{tab:addlabel}%
\end{table}%


% Table generated by Excel2LaTeX from sheet 'Sheet4'
\begin{table}[htbp]
  \centering
  \begin{tabular}{c|c|c|c|c|c}
    \hline
    \multicolumn{2}{c}{} & \textbf{S1} & \textbf{S2} & \textbf{S3} & \textbf{PC} \\
    \hline
    \multicolumn{2}{c|}{论文指标} & \textcolor{red}{0.488}  & 0.348  & \textcolor{red}{0.197}  & \textcolor{red}{0.360}  \\
    \hline
    \multirow{2}[0]{*}{复现指标} & epsilon-SVR & 0.5289  & \textcolor{red}{0.3319}  & 0.2473  & 0.2628  \\
    \cline{2-6}
    & nu-SVR & 0.5940  & 0.3488  & 0.2346  & 0.2146  \\
    \hline
  \end{tabular}%
  \caption{回归方案实验结果}
  \label{tab:addlabel}%
\end{table}%


% Table generated by Excel2LaTeX from sheet 'Sheet4'
\begin{table}[htbp]
  \centering
  \begin{tabular}{c|c|c|c|c|c}
    \hline
    &       & \textbf{Accurary} & \textbf{Precision} & \textbf{Recall} & \textbf{F1-score} \\
    \hline
    \multirow{5}[0]{*}{c-SVC} & \textbf{score=2.0} & 0.5193 & 0.0000  & 0.0000  & 0.0000 \\
    \cline{2-6}
    & \textbf{score=2.5} & - & 0.0000  & 0.0000  & 0.0000  \\
    \cline{2-6}
    & \textbf{score=3.0} & - & 0.0379  & 0.1158 & 0.0571 \\
    \cline{2-6}
    & \textbf{score=3.5} & - & 0.0000  & 0.0000  & 0.0000  \\
    \cline{2-6}
    & \textbf{score=4.0} & - & 0.5557  & 0.9511  & 0.6934  \\
    \hline
    nu-SVC &    -   &   -    &   -    &   -    & - \\
    \hline
  \end{tabular}%
  \caption{五分类(2.0/2.5/3.0/3.5/4.0)实验结果}
  \label{tab:addlabel}%
\end{table}%


% Table generated by Excel2LaTeX from sheet 'Sheet4'
\begin{table}[htbp]
  \centering
  \begin{tabular}{c|c|c|c|c|c|c|c}
    \hline
    \multicolumn{2}{c}{} & \multicolumn{3}{|c|}{\textbf{离题(score=2.0-3.5)}} & \multicolumn{3}{c}{\textbf{切题(score=4.0)}} \\
    \hline
    & \textbf{Accuracy} & \textbf{Precision} & \textbf{Recall} & \textbf{F1-score} & \textbf{Precision} & \textbf{Recall} & \textbf{F1-score} \\
    \hline
    \textbf{c-SVC} & 0.5470  & 0.1180  & 0.1000  & 0.1083  & 0.5519  & 0.9468  & 0.6894  \\
    \hline
    \textbf{nu-SVC} & 0.5916  & 0.5712  & 0.5018  & 0.5319  & 0.6077  & 0.6725  & 0.6367  \\
    \hline
  \end{tabular}%
  \caption{二分类([2.0-3.5]/4.0)实验结果}
  \label{tab:addlabel}%
\end{table}%


\section{下周计划}
\begin{itemize}
\item [1.] [***] 使用baseline特征进行其他方案的验证
\item [2.] [***] 继续之前的神经网络分类的工作
\end{itemize}
%\bibliography{reference}
%\bibliographystyle{coling}
%\bibliography{coling2020}

\end{CJK}
\end{document}


% include your own bib file like this:


%\begin{thebibliography}{}

%\bibitem[\protect\citename{Aho and Ullman}1972]{Aho:72}
%Alfred~V. Aho and Jeffrey~D. Ullman.
%\newblock 1972.
%\newblock {\em The Theory of Parsing, Translation and Compiling}, volume~1.
%\newblock Prentice-{Hall}, Englewood Cliffs, NJ.

%\bibitem[\protect\citename{{American Psychological Association}}1983]{APA:83}
%{American Psychological Association}.
%\newblock 1983.
%\newblock {\em Publications Manual}.
%\newblock American Psychological Association, Washington, DC.

%\bibitem[\protect\citename{{Association for Computing Machinery}}1983]{ACM:83}
%{Association for Computing Machinery}.
%\newblock 1983.
%\newblock {\em Computing Reviews}, 24(11):503--512.

%\bibitem[\protect\citename{Chandra \bgroup et al.\egroup }1981]{Chandra:81}
%Ashok~K. Chandra, Dexter~C. Kozen, and Larry~J. Stockmeyer.
%\newblock 1981.
%\newblock Alternation.
%\newblock {\em Journal of the Association for Computing Machinery},
%  28(1):114--133.

%\bibitem[\protect\citename{Gusfield}1997]{Gusfield:97}
%Dan Gusfield.
%\newblock 1997.
%\newblock {\em Algorithms on Strings, Trees and Sequences}.
%\newblock Cambridge University Press, Cambridge, UK.

%\bibitem[\protect\citename{Rasooli and Tetreault}2015]{rasooli-tetrault-2015}
%Mohammad~Sadegh Rasooli and Joel~R. Tetreault. 2015.
%\newblock {Yara parser: {A} fast and accurate dependency parser}.
%\newblock \emph{Computing Research Repository}, arXiv:1503.06733.
%\newblock Version 2.

%\bibitem[\protect\citename{Borschinger and Johnson}2011]{borsch2011}
%Benjamin Borschinger and Mark Johnson. 2011.
%\newblock A particle filter algorithm for {B}ayesian wordsegmentation.
%\newblock In \emph{Proceedings of the Australasian Language Technology Association %Workshop 2011}, pages 10--18, Canberra, Australia.

%\end{thebibliography}

