%
% File coling2020.tex
%
% Contact: feiliu@cs.ucf.edu & liang.huang.sh@gmail.com
%% Based on the style files for COLING-2018, which were, in turn,
%% Based on the style files for COLING-2016, which were, in turn,
%% Based on the style files for COLING-2014, which were, in turn,
%% Based on the style files for ACL-2014, which were, in turn,
%% Based on the style files for ACL-2013, which were, in turn,
%% Based on the style files for ACL-2012, which were, in turn,
%% based on the style files for ACL-2011, which were, in turn, 
%% based on the style files for ACL-2010, which were, in turn, 
%% based on the style files for ACL-IJCNLP-2009, which were, in turn,
%% based on the style files for EACL-2009 and IJCNLP-2008...

%% Based on the style files for EACL 2006 by 
%%e.agirre@ehu.es or Sergi.Balari@uab.es
%% and that of ACL 08 by Joakim Nivre and Noah Smith

\documentclass[11pt]{article}
\usepackage{coling2020}
\usepackage{times}
\usepackage{url}
\usepackage{latexsym}
\usepackage{indentfirst}

\usepackage{times}
\usepackage{latexsym}
\usepackage{times}
\usepackage{soul}
\usepackage{url}
\usepackage{amsmath}
\usepackage{amsthm}
\usepackage{booktabs}
\usepackage{algorithm}
\usepackage{algorithmic}
\usepackage{amssymb}
\usepackage{longtable}
\usepackage{graphicx}
\usepackage{CJK}
\usepackage{multirow}
\usepackage{color}

%\setlength\titlebox{5cm}
\colingfinalcopy % Uncomment this line for the final submission

% You can expand the titlebox if you need extra space
% to show all the authors. Please do not make the titlebox
% smaller than 5cm (the original size); we will check this
% in the camera-ready version and ask you to change it back.


\title{2021年03月29日进度汇报}

\author{屈原斌 \\
  首都师范大学 \\
    {\tt ybqu@cnu.edu.cn}}

\date{}

\begin{document}
\begin{CJK}{UTF8}{gkai}

\maketitle
\CJKindent
%\begin{abstract}

%\end{abstract}
\section{上周计划}


\begin{itemize}
  \item [1.] $Dataset_2$分别计算命题/半命题指标(未整理)
  \item [2.] 准备实验室开放基金项目结题报告
\end{itemize}


\section{上周计划执行情况}
\begin{itemize}
\item [1.] [$\checkmark$] 
\item [2.] [$\times$] 
\end{itemize}

\section{本周部分重点工作详述}

\subsection{$Dataset_2$指标更新}

\begin{itemize}
  \item 数据集:
  \begin{itemize}
    \item 共25个题目(12个命题,13个半命题),1250篇作文
  \end{itemize}
  \item 方案一指标见表1、2
  \item 结论:
  \begin{itemize}
    \item 分类模型命题作文指标优于半命题作文
    \item ? 生成模型半命题作文指标优于命题作文(待分析)
  \end{itemize}
\end{itemize}

% Table generated by Excel2LaTeX from sheet 'Sheet2'
\begin{table}[htbp]\small
  \centering
    \begin{tabular}{cccccccccc}
      \hline
      \multicolumn{3}{c}{\multirow{2}[0]{*}{\textcolor[rgb]{ 1,  0,  0}{}}} & \multicolumn{1}{c}{\multirow{2}[0]{*}{Accurary}} & \multicolumn{3}{p{12.57em}}{离题} & \multicolumn{3}{p{12.57em}}{不离题} \\
      \multicolumn{3}{c}{}  &       & \multicolumn{1}{p{4.19em}}{Precision} & \multicolumn{1}{p{4.19em}}{Recall} & \multicolumn{1}{p{4.19em}}{F1-score} & \multicolumn{1}{p{4.19em}}{Precision} & \multicolumn{1}{p{4.19em}}{Recall} & \multicolumn{1}{p{4.19em}}{F1-score} \\
      \hline
      \multicolumn{1}{c}{\multirow{6}[0]{*}{\textbf{HABiLstm}}} & \multicolumn{1}{c}{\multirow{3}[0]{*}{\textbf{开发集}}} & \textbf{命题} & 0.6528  & 0.6292  & 0.9722  & 0.7462  & 0.5625  & 0.3333  & 0.3964  \\
      &       & \textbf{半命题} & 0.8205  & 0.8115  & 9.9744  & 0.8647  & 0.8269  & 0.6667  & 0.7044  \\
      &       & \textbf{全部} & 0.7400  & 0.7240  & 0.9733  & 0.8078  & 0.7000  & 0.5067  & 0.5566  \\
      \cline{2-10}
      & \multicolumn{1}{c}{\multirow{3}[0]{*}{\textbf{测试集}}} & \textbf{命题} & 0.6099  & 0.5802  & 0.9167  & 0.7001  & 0.7234  & 0.3030  & 0.3813  \\
      &       & \textbf{半命题} & 0.6049  & 0.5755  & 0.8846  & 0.6926  & 0.6370  & 0.3252  & 0.4164  \\
      &       & \textbf{全部} & 0.6073  & 0.5777  & 0.9000  & 0.6962  & 0.6785  & 0.3245  & 0.3995  \\
      \hline
      \multicolumn{1}{c}{\multirow{6}[0]{*}{\textbf{BERT\_CLS}}} & \multicolumn{1}{c}{\multirow{3}[0]{*}{\textbf{开发集}}} & \textbf{命题} & 0.9722  & 0.9792  & 0.9722  & 0.9714  & 0.9792  & 0.9722  & 0.9714  \\
      &       & \textbf{半命题} & 0.9231  & 0.9038  & 0.9744  & 0.9297  & 0.9808  & 0.8718  & 0.9121  \\
      &       & \textbf{全部} & 0.9467  & 0.9400  & 0.9733  & 0.9497  & 0.9800  & 0.9200  & 0.9406  \\
      \cline{2-10}
      & \multicolumn{1}{c}{\multirow{3}[0]{*}{\textbf{测试集}}} & \textbf{命题} & 0.8485  & 0.8385  & 0.8788  & 0.8491  & 0.8941  & 0.8182  & 0.8441  \\
      &       & \textbf{半命题} & 0.8077  & 0.7532  & 0.9406  & 0.8316  & 0.9303  & 0.6748  & 0.7701  \\
      &       & \textbf{全部} & 0.8273  & 0.7941  & 0.9109  & 0.8400  & 0.9129  & 0.7436  & 0.8056  \\
      \hline
      \multicolumn{1}{c}{\multirow{6}[0]{*}{\textbf{XLNet\_CLS}}} & \multicolumn{1}{c}{\multirow{3}[0]{*}{\textbf{开发集}}} & \textbf{命题} & 1.0000  & 1.0000  & 1.0000  & 1.0000  & 1.0000  & 1.0000  & 1.0000  \\
      &       & \textbf{半命题} & 0.9615  & 0.9423  & 1.0000  & 0.9670  & 1.0000  & 0.9231  & 0.9538  \\
      &       & \textbf{全部} & 0.9800  & 0.9700  & 1.0000  & 0.9829  & 1.0000  & 0.9600  & 0.9760  \\
      \cline{2-10}
      & \multicolumn{1}{c}{\multirow{3}[0]{*}{\textbf{测试集}}} & \textbf{命题} & 0.9356  & 0.9311  & 0.9545  & 0.9372  & 0.9607  & 0.9167  & 0.9316  \\
      &       & \textbf{半命题} & 0.8497  & 0.7857  & 0.9930  & 0.8731  & 0.9933  & 0.7063  & 0.8112  \\
      &       & \textbf{全部} & 0.8910  & 0.8555  & 0.9745  & 0.9038  & 0.9777  & 0.8073  & 0.8690  \\
      \hline
    \end{tabular}%
    \caption{分类模型命题/半命题指标($Dataset_2$)}
  \label{tab:addlabel}%
\end{table}%

% Table generated by Excel2LaTeX from sheet 'Sheet2'
\begin{table}[htbp]\small
  \centering
    \begin{tabular}{cccccccccc}
      \hline
      \multicolumn{3}{c}{\multirow{2}[0]{*}{\textcolor[rgb]{ 1,  0,  0}{}}} & \multicolumn{1}{c}{\multirow{2}[0]{*}{Accurary}} & \multicolumn{3}{p{12.57em}}{离题} & \multicolumn{3}{p{12.57em}}{不离题} \\
      \multicolumn{3}{c}{}  &       & \multicolumn{1}{p{4.19em}}{Precision} & \multicolumn{1}{p{4.19em}}{Recall} & \multicolumn{1}{p{4.19em}}{F1-score} & \multicolumn{1}{p{4.19em}}{Precision} & \multicolumn{1}{p{4.19em}}{Recall} & \multicolumn{1}{p{4.19em}}{F1-score} \\
      \hline
      \multicolumn{1}{c}{\multirow{6}[0]{*}{\textbf{lstmabs}}} & \multicolumn{1}{c}{\multirow{3}[0]{*}{\textbf{开发集}}} & \textbf{命题} & 0.8333  & 0.8083  & 0.9722  & 0.8662  & 0.8958  & 0.6945  & 0.7548  \\
      &       & \textbf{半命题} & 0.8461  & 0.8282  & 0.9487  & 0.8701  & 0.8782  & 0.7436  & 0.7864  \\
      &       & \textbf{全部} & 0.8400  & 0.8187  & 0.9600  & 0.8683  & 0.8867  & 0.7200  & 0.7712  \\
      \cline{2-10}
      & \multicolumn{1}{c}{\multirow{3}[0]{*}{\textbf{测试集}}} & \textbf{命题} & 0.7614  & 0.7352  & 0.8977  & 0.7900  & 0.8994  & 0.6250  & 0.6983  \\
      &       & \textbf{半命题} & 0.7552  & 0.7022  & 0.9301  & 0.7955  & 0.9108  & 0.5804  & 0.6773  \\
      &       & \textbf{全部} & 0.7582  & 0.7180  & 0.9145  & 0.7928  & 0.9053  & 0.6018  & 0.6874  \\
      \hline
      \multicolumn{1}{c}{\multirow{6}[0]{*}{\textbf{bertabs}}} & \multicolumn{1}{c}{\multirow{3}[0]{*}{\textbf{开发集}}} & \textbf{命题} & 0.8611  & 0.8292  & 1.0000  & 0.8948  & 0.9167  & 0.7222  & 0.7833  \\
      &       & \textbf{半命题} & 0.8590  & 0.8397  & 0.9744  & 0.8901  & 0.8205  & 0.7436  & 0.7744  \\
      &       & \textbf{全部} & 0.8600  & 0.8347  & 0.9867  & 0.8924  & 0.8667  & 0.7333  & 0.7787  \\
      \cline{2-10}
      & \multicolumn{1}{c}{\multirow{3}[0]{*}{\textbf{测试集}}} & \textbf{命题} & 0.7898  & 0.7749  & 0.8674  & 0.8027  & 0.8851  & 0.7121  & 0.7555  \\
      &       & \textbf{半命题} & 0.8007  & 0.7723  & 0.9336  & 0.8341  & 0.8462  & 0.6678  & 0.7254  \\
      &       & \textbf{全部} & 0.7955  & 0.7736  & 0.9018  & 0.8190  & 0.8648  & 0.6891  & 0.7398  \\
      \hline
    \end{tabular}%
    \caption{生成模型命题/半命题指标($Dataset_2$)}
  \label{tab:addlabel}%
\end{table}%


\section{下周计划}
\begin{itemize}
\item [1.] [***] 完成结题报告
\item [2.] [***] 完成baseline
\item [3.] [***] 英文模型表示分析
\end{itemize}
%\bibliography{reference}
%\bibliographystyle{coling}
%\bibliography{coling2020}

\end{CJK}
\end{document}


% include your own bib file like this:


%\begin{thebibliography}{}

%\bibitem[\protect\citename{Aho and Ullman}1972]{Aho:72}
%Alfred~V. Aho and Jeffrey~D. Ullman.
%\newblock 1972.
%\newblock {\em The Theory of Parsing, Translation and Compiling}, volume~1.
%\newblock Prentice-{Hall}, Englewood Cliffs, NJ.

%\bibitem[\protect\citename{{American Psychological Association}}1983]{APA:83}
%{American Psychological Association}.
%\newblock 1983.
%\newblock {\em Publications Manual}.
%\newblock American Psychological Association, Washington, DC.

%\bibitem[\protect\citename{{Association for Computing Machinery}}1983]{ACM:83}
%{Association for Computing Machinery}.
%\newblock 1983.
%\newblock {\em Computing Reviews}, 24(11):503--512.

%\bibitem[\protect\citename{Chandra \bgroup et al.\egroup }1981]{Chandra:81}
%Ashok~K. Chandra, Dexter~C. Kozen, and Larry~J. Stockmeyer.
%\newblock 1981.
%\newblock Alternation.
%\newblock {\em Journal of the Association for Computing Machinery},
%  28(1):114--133.

%\bibitem[\protect\citename{Gusfield}1997]{Gusfield:97}
%Dan Gusfield.
%\newblock 1997.
%\newblock {\em Algorithms on Strings, Trees and Sequences}.
%\newblock Cambridge University Press, Cambridge, UK.

%\bibitem[\protect\citename{Rasooli and Tetreault}2015]{rasooli-tetrault-2015}
%Mohammad~Sadegh Rasooli and Joel~R. Tetreault. 2015.
%\newblock {Yara parser: {A} fast and accurate dependency parser}.
%\newblock \emph{Computing Research Repository}, arXiv:1503.06733.
%\newblock Version 2.

%\bibitem[\protect\citename{Borschinger and Johnson}2011]{borsch2011}
%Benjamin Borschinger and Mark Johnson. 2011.
%\newblock A particle filter algorithm for {B}ayesian wordsegmentation.
%\newblock In \emph{Proceedings of the Australasian Language Technology Association %Workshop 2011}, pages 10--18, Canberra, Australia.

%\end{thebibliography}

