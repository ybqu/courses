%
% File coling2020.tex
%
% Contact: feiliu@cs.ucf.edu & liang.huang.sh@gmail.com
%% Based on the style files for COLING-2018, which were, in turn,
%% Based on the style files for COLING-2016, which were, in turn,
%% Based on the style files for COLING-2014, which were, in turn,
%% Based on the style files for ACL-2014, which were, in turn,
%% Based on the style files for ACL-2013, which were, in turn,
%% Based on the style files for ACL-2012, which were, in turn,
%% based on the style files for ACL-2011, which were, in turn, 
%% based on the style files for ACL-2010, which were, in turn, 
%% based on the style files for ACL-IJCNLP-2009, which were, in turn,
%% based on the style files for EACL-2009 and IJCNLP-2008...

%% Based on the style files for EACL 2006 by 
%%e.agirre@ehu.es or Sergi.Balari@uab.es
%% and that of ACL 08 by Joakim Nivre and Noah Smith

\documentclass[11pt]{article}
\usepackage{coling2020}
\usepackage{times}
\usepackage{url}
\usepackage{latexsym}
\usepackage{indentfirst}

\usepackage{times}
\usepackage{latexsym}
\usepackage{times}
\usepackage{soul}
\usepackage{url}
\usepackage{amsmath}
\usepackage{amsthm}
\usepackage{booktabs}
\usepackage{algorithm}
\usepackage{algorithmic}
\usepackage{amssymb}
\usepackage{longtable}
\usepackage{graphicx}
\usepackage{CJK}
\usepackage{multirow}
\usepackage{color}

%\setlength\titlebox{5cm}
\colingfinalcopy % Uncomment this line for the final submission

% You can expand the titlebox if you need extra space
% to show all the authors. Please do not make the titlebox
% smaller than 5cm (the original size); we will check this
% in the camera-ready version and ask you to change it back.


\title{周报,2021年04月19日}

\author{屈原斌 \\
  首都师范大学 \\
    {\tt ybqu@cnu.edu.cn}}

\date{}

\begin{document}
\begin{CJK}{UTF8}{gkai}

\maketitle
\CJKindent
%\begin{abstract}

%\end{abstract}
\section{上周计划}


\begin{itemize}
  \item [1.] 更新baseline相关实验
\end{itemize}


\section{上周计划执行情况}
\begin{itemize}
\item [1.] [$\checkmark$] 
\end{itemize}

\section{本周部分重点工作详述}

\subsection{中文离题更新}

\begin{itemize}
  \item 数据集:
  \begin{itemize}
    \item 中文数据集,离题:不离题=1:39
  \end{itemize}
  \item 优化:使用全部测试集作文对生成模型继续训练
  \item 指标:见表1、表2
  \item 结论:
  \begin{itemize}
    \item 使用测试集数据微调后不离题指标下降,方案更容易判断为离题
  \end{itemize}
\end{itemize}

% Table generated by Excel2LaTeX from sheet '中文数据集-Dataset1'
\begin{table}[htbp]\small
  \centering
  \begin{tabular}{cccccccc}
    \hline
    \multicolumn{2}{c}{\multirow{2}[0]{*}{}} & \multicolumn{3}{p{16.94em}}{\textbf{离题}} & \multicolumn{3}{p{12.57em}}{\textbf{不离题}} \\
    \multicolumn{2}{c}{} & \multicolumn{1}{p{5.625em}}{\textbf{Precision}} & \multicolumn{1}{p{5.375em}}{\textbf{Recall}} & \multicolumn{1}{p{5.94em}}{\textbf{F1-score}} & \multicolumn{1}{p{4.19em}}{\textbf{Precision}} & \multicolumn{1}{p{4.19em}}{\textbf{Recall}} & \multicolumn{1}{p{4.19em}}{\textbf{F1-score}} \\
    \hline
    \multicolumn{1}{c}{\multirow{2}[0]{*}{\textbf{BERT}}} & \textbf{开发集} & 0.2174  & 0.5688  & 0.2358  & 0.9827  & 0.8033  & 0.8644  \\
    & \textbf{测试集} & 0.0980  & 0.3615  & 0.1303  & 0.9703  & 0.8021  & 0.8599  \\
    \hline
    \multicolumn{1}{c}{\multirow{2}[0]{*}{\textbf{BERT(测试集上继续训练)}}} & \textbf{开发集} & 0.2016  & 0.7929  & 0.2894  & 0.9483  & 0.4710  & 0.5852  \\
    & \textbf{测试集} & 0.1472  & 0.5638  & 0.2025  & 0.8790  & 0.4406  & 0.5420  \\
    \hline
  \end{tabular}%
  \caption{方案一指标更新}
  \label{tab:addlabel}%
\end{table}%

% Table generated by Excel2LaTeX from sheet '中文数据集-Dataset1'
\begin{table}[htbp]\small
  \centering
    \begin{tabular}{cccccccc}
      \hline
      \multicolumn{2}{c}{\multirow{2}[0]{*}{}} & \multicolumn{3}{p{12.57em}}{\textbf{离题}} & \multicolumn{3}{p{12.57em}}{\textbf{不离题}} \\
      \multicolumn{2}{c}{} & \multicolumn{1}{p{4.19em}}{\textbf{Precision}} & \multicolumn{1}{p{4.19em}}{\textbf{Recall}} & \multicolumn{1}{p{4.19em}}{\textbf{F1-score}} & \multicolumn{1}{p{4.19em}}{\textbf{Precision}} & \multicolumn{1}{p{4.19em}}{\textbf{Recall}} & \multicolumn{1}{p{4.19em}}{\textbf{F1-score}} \\
      \hline
      \multicolumn{1}{c}{\multirow{2}[0]{*}{\textbf{BERT\_ABS\_CLS}}} & \textbf{开发集} & 0.2566  & 0.6328  & 0.1869  & 0.8154  & 0.5790  & 0.6293  \\
      & \textbf{测试集} & 0.1071  & 0.2383  & 0.0969  & 0.8071  & 0.5741  & 0.6251  \\
      \hline
      \multicolumn{1}{c}{\multirow{2}[0]{*}{\textbf{BERT\_ABS\_测试集上继续训练}}} & \textbf{开发集} & 0.1353  & 1.0000  & 0.2231  & 0.0000  & 0.0000  & 0.0000  \\
      & \textbf{测试集} & 0.1430  & 1.0000  & 0.2344  & 0.0000  & 0.0000  & 0.0000  \\
      \hline
    \end{tabular}%
    \caption{方案二指标更新}
  \label{tab:addlabel}%
\end{table}%


\section{下周计划}
\begin{itemize}
\item [1.] [***] 标题-正文匹配实验
\item [2.] [***] 更新离题实验
\end{itemize}
%\bibliography{reference}
%\bibliographystyle{coling}
%\bibliography{coling2020}

\end{CJK}
\end{document}


% include your own bib file like this:


%\begin{thebibliography}{}

%\bibitem[\protect\citename{Aho and Ullman}1972]{Aho:72}
%Alfred~V. Aho and Jeffrey~D. Ullman.
%\newblock 1972.
%\newblock {\em The Theory of Parsing, Translation and Compiling}, volume~1.
%\newblock Prentice-{Hall}, Englewood Cliffs, NJ.

%\bibitem[\protect\citename{{American Psychological Association}}1983]{APA:83}
%{American Psychological Association}.
%\newblock 1983.
%\newblock {\em Publications Manual}.
%\newblock American Psychological Association, Washington, DC.

%\bibitem[\protect\citename{{Association for Computing Machinery}}1983]{ACM:83}
%{Association for Computing Machinery}.
%\newblock 1983.
%\newblock {\em Computing Reviews}, 24(11):503--512.

%\bibitem[\protect\citename{Chandra \bgroup et al.\egroup }1981]{Chandra:81}
%Ashok~K. Chandra, Dexter~C. Kozen, and Larry~J. Stockmeyer.
%\newblock 1981.
%\newblock Alternation.
%\newblock {\em Journal of the Association for Computing Machinery},
%  28(1):114--133.

%\bibitem[\protect\citename{Gusfield}1997]{Gusfield:97}
%Dan Gusfield.
%\newblock 1997.
%\newblock {\em Algorithms on Strings, Trees and Sequences}.
%\newblock Cambridge University Press, Cambridge, UK.

%\bibitem[\protect\citename{Rasooli and Tetreault}2015]{rasooli-tetrault-2015}
%Mohammad~Sadegh Rasooli and Joel~R. Tetreault. 2015.
%\newblock {Yara parser: {A} fast and accurate dependency parser}.
%\newblock \emph{Computing Research Repository}, arXiv:1503.06733.
%\newblock Version 2.

%\bibitem[\protect\citename{Borschinger and Johnson}2011]{borsch2011}
%Benjamin Borschinger and Mark Johnson. 2011.
%\newblock A particle filter algorithm for {B}ayesian wordsegmentation.
%\newblock In \emph{Proceedings of the Australasian Language Technology Association %Workshop 2011}, pages 10--18, Canberra, Australia.

%\end{thebibliography}

