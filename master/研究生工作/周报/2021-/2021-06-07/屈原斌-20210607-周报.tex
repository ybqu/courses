%
% File coling2020.tex
%
% Contact: feiliu@cs.ucf.edu & liang.huang.sh@gmail.com
%% Based on the style files for COLING-2018, which were, in turn,
%% Based on the style files for COLING-2016, which were, in turn,
%% Based on the style files for COLING-2014, which were, in turn,
%% Based on the style files for ACL-2014, which were, in turn,
%% Based on the style files for ACL-2013, which were, in turn,
%% Based on the style files for ACL-2012, which were, in turn,
%% based on the style files for ACL-2011, which were, in turn, 
%% based on the style files for ACL-2010, which were, in turn, 
%% based on the style files for ACL-IJCNLP-2009, which were, in turn,
%% based on the style files for EACL-2009 and IJCNLP-2008...

%% Based on the style files for EACL 2006 by 
%%e.agirre@ehu.es or Sergi.Balari@uab.es
%% and that of ACL 08 by Joakim Nivre and Noah Smith

\documentclass[11pt]{article}
\usepackage{coling2020}
\usepackage{times}
\usepackage{url}
\usepackage{latexsym}
\usepackage{indentfirst}

\usepackage{times}
\usepackage{latexsym}
\usepackage{times}
\usepackage{soul}
\usepackage{url}
\usepackage{amsmath}
\usepackage{amsthm}
\usepackage{booktabs}
\usepackage{algorithm}
\usepackage{algorithmic}
\usepackage{amssymb}
\usepackage{longtable}
\usepackage{graphicx}
\usepackage{CJK}
\usepackage{multirow}
\usepackage{color}

%\setlength\titlebox{5cm}
\colingfinalcopy % Uncomment this line for the final submission

% You can expand the titlebox if you need extra space
% to show all the authors. Please do not make the titlebox
% smaller than 5cm (the original size); we will check this
% in the camera-ready version and ask you to change it back.


\title{周报,2021年06月07日}

\author{屈原斌 \\
  首都师范大学 \\
    {\tt ybqu@cnu.edu.cn}}

\date{}

\begin{document}
\begin{CJK}{UTF8}{gkai}

\maketitle
\CJKindent
%\begin{abstract}

%\end{abstract}
\section{上周计划}


\begin{itemize}
  \item [1.] 标题-正文匹配实验更新
  \item [2.] 初中场景检测结果分析
  \item [3.] 写Demo
  \item [4.] 格式检查
\end{itemize}


\section{上周计划执行情况}
\begin{itemize}
\item [1.] [$\times$] 部分实验未完成
\item [2.] [$\times$] 
\item [3.] [$\times$] 
\end{itemize}

\section{本周部分重点工作详述}

\subsection{标题-正文匹配实验更新}

\begin{itemize}
  \item 数据集:
  \begin{itemize}
    \item 同步作文,共59个主题,199篇原始作文,208篇构造数据
    \item 数据比例-不离题:部分离题:完全离题=202:127:78
  \end{itemize}
  \item 实验策略:
  \begin{itemize}
    \item [1.] baseline:使用n-gram和关键词计算相似度
    \item [2.] 使用bert生成模型获取表示,检索相应标题,计算相似度
    \begin{itemize}
      \item 划分验证集调参
    \end{itemize}
    % \item [!] 自监督方案
  \end{itemize}
  \item 指标见表1
  \item 结论:
  \begin{itemize}
    \item 参考作文使用同步作文结果优于乐乐课堂
    \item 使用whitening之后指标下降比较多
    \item k值增大指标下降
  \end{itemize}
\end{itemize}

% Table generated by Excel2LaTeX from sheet 'Sheet1'
\begin{table}[htbp]
  \centering
  \resizebox*{\textwidth}{!}{
    \begin{tabular}{ccccccccccccccc}
      \hline
      \multicolumn{3}{c}{\textcolor[rgb]{ 1,  1,  1}{}} & \multicolumn{3}{c}{\textbf{人1-人2}} & \multicolumn{3}{c}{\textbf{人1-机}} & \multicolumn{3}{c}{\textbf{人2-机}} & \multicolumn{3}{c}{\textbf{仲裁-机}} \\
      \hline
      \multicolumn{3}{c}{}  & \multicolumn{1}{c}{\textbf{ACC}} & \multicolumn{1}{c}{\textbf{Marco-F1}} & \multicolumn{1}{c}{\textbf{跨二档率}} & \multicolumn{1}{c}{\textbf{ACC}} & \multicolumn{1}{c}{\textbf{Marco-F1}} & \multicolumn{1}{c}{\textbf{跨二档率}} & \multicolumn{1}{c}{\textbf{ACC}} & \multicolumn{1}{c}{\textbf{Marco-F1}} & \multicolumn{1}{c}{\textbf{跨二档率}} & \multicolumn{1}{c}{\textbf{ACC}} & \multicolumn{1}{c}{\textbf{Marco-F1}} & \multicolumn{1}{c}{\textbf{跨二档率}} \\
      \hline
      \multicolumn{3}{p{12.705em}}{\textbf{Baseline}} & 0.6855 & 0.6265 & 0.0172 & 0.4545 & 0.4282 & 0.1081 & 0.4201 & 0.392 & 0.14  & 0.4619 & 0.4388 & 0.113 \\
      \hline
      \multicolumn{1}{r}{\multirow{8}[0]{*}{\textbf{参考作文:乐乐课堂}}} & \multicolumn{1}{c}{\multirow{4}[0]{*}{\textbf{BERT\_Gen\_CLS}}} & \textbf{Topk=5} &       &       &       & 0.4251 & 0.3657 & 0.1499 & 0.4398 & 0.3696 & 0.172 & 0.4644 & 0.4032 & 0.1523 \\
      &       & \textbf{Topk=10} &       &       &       & 0.4373 & 0.3318 & 0.1474 & 0.4717 & 0.3618 & 0.1843 & 0.4717 & 0.3619 & 0.1548 \\
      &       & \textbf{Topk=20} &       &       &       & 0.4324 & 0.3096 & 0.1622 & 0.4816 & 0.3424 & 0.199 & 0.4619 & 0.3237 & 0.1695 \\
      &       & \textbf{Topk=50} &       &       &       & 0.4324 & 0.2922 & 0.1425 & 0.4644 & 0.3025 & 0.1818 & 0.4447 & 0.275 & 0.1474 \\
      & \multicolumn{1}{c}{\multirow{4}[0]{*}{\textbf{+whitening}}} & \textbf{Topk=5} &       &       &       & 0.2531 & 0.2435 & 0.317 & 0.226 & 0.2185 & 0.3563 & 0.2383 & 0.2334 & 0.3489 \\
      &       & \textbf{Topk=10} &       &       &       & 0.2482 & 0.24  & 0.3145 & 0.231 & 0.2219 & 0.3636 & 0.2383 & 0.2335 & 0.3514 \\
      &       & \textbf{Topk=20} &       &       &       & 0.3219 & 0.2986 & 0.2334 & 0.2727 & 0.263 & 0.2703 & 0.2899 & 0.2757 & 0.2482 \\
      &       & \textbf{Topk=50} &       &       &       & 0.3145 & 0.2824 & 0.2187 & 0.2973 & 0.2798 & 0.2629 & 0.2948 & 0.2711 & 0.2408 \\
      \hline
      \multicolumn{1}{r}{\multirow{8}[0]{*}{\textbf{参考作文:同步作文}}} & \multicolumn{1}{c}{\multirow{4}[0]{*}{\textbf{BERT\_Gen\_CLS}}} & \textbf{Topk=5} &       &       &       & 0.4889 & 0.4519 & 0.0885 & 0.5037 & 0.4595 & 0.1253 & 0.5479 & 0.5146 & 0.1057 \\
      &       & \textbf{Topk=10} &       &       &       & 0.5381 & 0.473 & 0.0958 & 0.5332 & 0.4577 & 0.1327 & 0.5602 & 0.4921 & 0.1057 \\
      &       & \textbf{Topk=20} &       &       &       & 0.4865 & 0.4357 & 0.0762 & 0.5209 & 0.4665 & 0.1278 & 0.5356 & 0.4862 & 0.0934 \\
      &       & \textbf{Topk=50} &       &       &       & 0.5135 & 0.4567 & 0.0909 & 0.5209 & 0.4535 & 0.14  & 0.5258 & 0.461 & 0.1007 \\
      & \multicolumn{1}{c}{\multirow{4}[0]{*}{\textbf{+whitening}}} & \textbf{Topk=5} &       &       &       & 0.2432 & 0.2346 & 0.3342 & 0.2727 & 0.2558 & 0.3759 & 0.2506 & 0.24  & 0.3587 \\
      &       & \textbf{Topk=10} &       &       &       & 0.2948 & 0.2696 & 0.2383 & 0.2555 & 0.2467 & 0.2703 & 0.2776 & 0.263 & 0.2629 \\
      &       & \textbf{Topk=20} &       &       &       & 0.3145 & 0.2982 & 0.2138 & 0.2899 & 0.2882 & 0.2457 & 0.3022 & 0.2958 & 0.2236 \\
      &       & \textbf{Topk=50} &       &       &       & 0.317 & 0.2849 & 0.1966 & 0.3145 & 0.3032 & 0.2211 & 0.312 & 0.2923 & 0.2039 \\
      \hline
    \end{tabular}}%
  \caption{Add caption}
  \label{tab:addlabel}%
\end{table}%


\section{下周计划}
\begin{itemize}
\item [1.] [***] 完成Demo
\item [2.] [***] 标题-正文匹配分命题/半命题更新指标
% \item [3.] [***] 调研自监督方法,尝试构建负样本
\end{itemize}
%\bibliography{reference}
%\bibliographystyle{coling}
%\bibliography{coling2020}

\end{CJK}
\end{document}


% include your own bib file like this:


%\begin{thebibliography}{}

%\bibitem[\protect\citename{Aho and Ullman}1972]{Aho:72}
%Alfred~V. Aho and Jeffrey~D. Ullman.
%\newblock 1972.
%\newblock {\em The Theory of Parsing, Translation and Compiling}, volume~1.
%\newblock Prentice-{Hall}, Englewood Cliffs, NJ.

%\bibitem[\protect\citename{{American Psychological Association}}1983]{APA:83}
%{American Psychological Association}.
%\newblock 1983.
%\newblock {\em Publications Manual}.
%\newblock American Psychological Association, Washington, DC.

%\bibitem[\protect\citename{{Association for Computing Machinery}}1983]{ACM:83}
%{Association for Computing Machinery}.
%\newblock 1983.
%\newblock {\em Computing Reviews}, 24(11):503--512.

%\bibitem[\protect\citename{Chandra \bgroup et al.\egroup }1981]{Chandra:81}
%Ashok~K. Chandra, Dexter~C. Kozen, and Larry~J. Stockmeyer.
%\newblock 1981.
%\newblock Alternation.
%\newblock {\em Journal of the Association for Computing Machinery},
%  28(1):114--133.

%\bibitem[\protect\citename{Gusfield}1997]{Gusfield:97}
%Dan Gusfield.
%\newblock 1997.
%\newblock {\em Algorithms on Strings, Trees and Sequences}.
%\newblock Cambridge University Press, Cambridge, UK.

%\bibitem[\protect\citename{Rasooli and Tetreault}2015]{rasooli-tetrault-2015}
%Mohammad~Sadegh Rasooli and Joel~R. Tetreault. 2015.
%\newblock {Yara parser: {A} fast and accurate dependency parser}.
%\newblock \emph{Computing Research Repository}, arXiv:1503.06733.
%\newblock Version 2.

%\bibitem[\protect\citename{Borschinger and Johnson}2011]{borsch2011}
%Benjamin Borschinger and Mark Johnson. 2011.
%\newblock A particle filter algorithm for {B}ayesian wordsegmentation.
%\newblock In \emph{Proceedings of the Australasian Language Technology Association %Workshop 2011}, pages 10--18, Canberra, Australia.

%\end{thebibliography}

