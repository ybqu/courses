%
% File coling2020.tex
%
% Contact: feiliu@cs.ucf.edu & liang.huang.sh@gmail.com
%% Based on the style files for COLING-2018, which were, in turn,
%% Based on the style files for COLING-2016, which were, in turn,
%% Based on the style files for COLING-2014, which were, in turn,
%% Based on the style files for ACL-2014, which were, in turn,
%% Based on the style files for ACL-2013, which were, in turn,
%% Based on the style files for ACL-2012, which were, in turn,
%% based on the style files for ACL-2011, which were, in turn, 
%% based on the style files for ACL-2010, which were, in turn, 
%% based on the style files for ACL-IJCNLP-2009, which were, in turn,
%% based on the style files for EACL-2009 and IJCNLP-2008...

%% Based on the style files for EACL 2006 by 
%%e.agirre@ehu.es or Sergi.Balari@uab.es
%% and that of ACL 08 by Joakim Nivre and Noah Smith

\documentclass[11pt]{article}
\usepackage{coling2020}
\usepackage{times}
\usepackage{url}
\usepackage{latexsym}
\usepackage{indentfirst}

\usepackage{times}
\usepackage{latexsym}
\usepackage{times}
\usepackage{soul}
\usepackage{url}
\usepackage{amsmath}
\usepackage{amsthm}
\usepackage{booktabs}
\usepackage{algorithm}
\usepackage{algorithmic}
\usepackage{amssymb}
\usepackage{longtable}
\usepackage{graphicx}
\usepackage{CJK}
\usepackage{multirow}
\usepackage{color}

%\setlength\titlebox{5cm}
\colingfinalcopy % Uncomment this line for the final submission

% You can expand the titlebox if you need extra space
% to show all the authors. Please do not make the titlebox
% smaller than 5cm (the original size); we will check this
% in the camera-ready version and ask you to change it back.


\title{周报,2021年06月21日}

\author{屈原斌 \\
  首都师范大学 \\
    {\tt ybqu@cnu.edu.cn}}

\date{}

\begin{document}
\begin{CJK}{UTF8}{gkai}

\maketitle
\CJKindent
%\begin{abstract}

%\end{abstract}
\section{上周计划}


\begin{itemize}
  \item [1.] 标题-正文匹配实验更新
  \item [2.] 梳理技术方案图
  % \item [3.] 写Demo
  \item [3.] 格式检查
\end{itemize}


\section{上周计划执行情况}
\begin{itemize}
\item [1.] [$\times$] 
\item [2.] [$\times$] 
\item [3.] [$\times$] 
\end{itemize}

\section{本周部分重点工作详述}

\subsection{标题-正文匹配实验更新}

\begin{itemize}
  \item 数据集:
  \begin{itemize}
    \item $Dataset_1$: 同步作文,共59个主题,199篇原始作文,208篇构造数据,不离题:部分离题:完全离题=202:127:78
    \item $Dataset_2$: 乐乐课堂随机替换题目,共2200篇,不离题:完全离题2000:200
  \end{itemize}
  \item 实验策略:
  \begin{itemize}
    \item [1.] 策略一:
    \begin{itemize}
      \item 使用bert生成模型获取表示,检索相应标题,计算相似度
      \begin{itemize}
        \item baseline:使用TFIDF和主题词计算相似度
      \end{itemize}
    \end{itemize} 
    \begin{itemize}
      \item 参考作文:50W 乐乐课堂作文
      \item 指标见表1,2
      \item 结论:
      \begin{itemize}
        \item 指标与参考作文数成正比
        \item $Dataset_1$CLS指标优于Last1avg, $Dataset_2$相反
        \item $Dataset_1$相似度取max指标最优,$Dataset_2$相似度取平均指标最优
      \end{itemize}
    \end{itemize}
    \item [2.] 策略二:
    \begin{itemize}
      \item 使用BertNSP做二分类任务
      \item 指标见表3
      \item 结论:
      \begin{itemize}
        \item 
      \end{itemize}
    \end{itemize}
    % \item [!] 自监督方案
  \end{itemize}
\end{itemize}

% Table generated by Excel2LaTeX from sheet 'Sheet4'
\begin{table}[htbp]
  \centering
  \resizebox*{\textwidth}{!}{
  \begin{tabular}{c|ccccccccccccccc}
    \hline
    \multicolumn{3}{c}{}  & \multicolumn{1}{c}{\multirow{2}[0]{*}{阈值}} & \multirow{2}[0]{*}{Acc} & \multicolumn{3}{c}{优} & \multicolumn{3}{c}{良} & \multicolumn{3}{c}{中} & \multirow{2}[0]{*}{Marco-F1} & \multirow{2}[0]{*}{跨二档率} \\
    \multicolumn{3}{c}{}  &       &       & P     & R     & F1-score & P     & R     & F1-score & P     & R     & F1-score &       &  \\
    \hline
    \multicolumn{3}{c}{baseline} & \multicolumn{1}{c}{0.75-0.85} & 0.4767 & 0.5547 & 0.7277 & 0.6296 & 0.3236 & 0.2362 & 0.2727 & 0.3469 & 0.2179 & 0.2677 & 0.39  & 0.1302 \\
    \hline
    \multirow{6}[0]{*}{BERT-Gen} & \multicolumn{1}{c}{\multirow{2}[0]{*}{topk表示取平均计算相似度}} & CLS   & topk=50,0.50-0.60 & 0.516 & 0.5992 & 0.7624 & 0.671 & 0.4253 & 0.2913 & 0.3458 & 0.3016 & 0.2436 & 0.2695 & 0.4288 & 0.14 \\
    &       & Last1avg & topk=50,0.50-0.60 & 0.4963 & 0.5869 & 0.7525 & 0.6594 & 0.3735 & 0.2441 & 0.2952 & 0.2923 & 0.2436 & 0.2657 & 0.4068 & 0.14 \\
    \cline{2-16}
    & \multirow{2}[0]{*}{topk相似度取max} & CLS   & topk=50,0.60-0.80 & 0.4914 & 0.6685 & 0.6089 & 0.6373 & 0.3548 & 0.3465 & 0.3506 & 0.3333 & 0.4231 & 0.3729 & 0.4536 & 0.3729 \\
    &       & Last1avg & topk=100,0.60-0.80 & 0.5037 & 0.6635 & 0.6832 & 0.6732 & 0.3566 & 0.3622 & 0.3594 & 0.3   & 0.2692 & 0.2838 & 0.4388 & 0.0934 \\
    \cline{2-16}
    & \multirow{2}[0]{*}{topk相似度取平均} & CLS   & topk=20,0.35-0.50 & 0.4816 & 0.6444 & 0.5743 & 0.6073 & 0.3713 & 0.4882 & 0.4218 & 0.3   & 0.2308 & 0.2609 & 0.43  & 0.1007 \\
    &       & Last1avg & topk=20,0.35-0.45 & 0.4889 & 0.5926 & 0.7129 & 0.6472 & 0.3786 & 0.3071 & 0.3391 & 0.2623 & 0.2051 & 0.2302 & 0.4055 & 0.1376 \\
    \hline
  \end{tabular}}%
  \caption{$Dataset_1$指标更新}
  \label{tab:addlabel}%
\end{table}%

% Table generated by Excel2LaTeX from sheet 'Sheet4'
\begin{table}[htbp]
  \centering
  \resizebox*{\textwidth}{!}{
  \begin{tabular}{cccccccccc}
    \hline
    \multicolumn{2}{c}{} & \multirow{2}[0]{*}{阈值} &       & 优     &       &       & 中     &       &  \\
    \multicolumn{2}{c}{} &       & Accuracy & Precision & Recall & F1-score & Precision & Recall & F1-score \\
    \hline
    \multicolumn{1}{c}{\multirow{2}[0]{*}{topk\_取平均计算相似度}} & CLS   & topk=15,0.55-0.65 & 0.8664 & 0.9677 & 0.8825 & 0.9231 & 0.375 & 0.705 & 0.4896 \\
    & Last1avg & topk=5,0.50-0.60 & 0.8614 & 0.9796 & 0.8655 & 0.919 & 0.3788 & 0.82  & 0.5182 \\
    \multirow{2}[0]{*}{topk相似度取max} & CLS   & topk=5,0.55-0.65 & 0.8673 & 0.9809 & 0.871 & 0.9227 & 0.3915 & 0.83  & 0.5321 \\
    & Last1avg & topk=5,0.55-0.65 & 0.8814 & 0.9801 & 0.8875 & 0.9315 & 0.4216 & 0.82  & 0.5569 \\
    \multirow{2}[0]{*}{topk相似度取平均} & CLS   & topk=50,0.65-0.75 & 0.9018 & 0.9822 & 0.9085 & 0.9439 & 0.4771 & 0.835 & 0.6073 \\
    & Last1avg & topk=100,0.70-0.80 & 0.9095 & 0.9818 & 0.9175 & 0.9486 & 0.5015 & 0.83  & 0.6262 \\
    \hline
  \end{tabular}}%
  \caption{$Dataset_2$指标更新}
  \label{tab:addlabel}%
\end{table}%


% Table generated by Excel2LaTeX from sheet 'Sheet4'
\begin{table}[htbp]
  \centering
  \begin{tabular}{cccccccc}
    \hline
    \multicolumn{2}{c}{\textcolor[rgb]{ 1,  1,  1}{}} & \multicolumn{3}{c}{\textbf{优}} & \multicolumn{3}{c}{\textbf{中}} \\
    \multicolumn{1}{c}{} & \multicolumn{1}{c}{\textbf{Acc}} & \multicolumn{1}{c}{\textbf{Precision}} & \multicolumn{1}{c}{\textbf{Recall}} & \multicolumn{1}{c}{\textbf{F1-score}} & \multicolumn{1}{c}{\textbf{Precision}} & \multicolumn{1}{c}{\textbf{Recall}} & \multicolumn{1}{c}{\textbf{F1-score}} \\
    \hline
    \textbf{Baseline} & 0.8814 & 0.9801 & 0.8875 & 0.9315 & 0.4216 & 0.82  & 0.5569 \\
    \hline
    \textbf{训练集15W} & 0.9718 & 0.9929 & 0.976 & 0.9844 & 0.7949 & 0.93  & \textbf{0.8571} \\
    \textbf{训练集30W} & 0.9359 & 0.9952 & 0.934 & 0.9636 & 0.5913 & 0.955 & 0.7304 \\
    \textbf{训练集60W} & 0.9227 & 0.9962 & 0.9185 & 0.9558 & 0.5421 & 0.965 & 0.6942 \\
    \textbf{训练集75W} & 0.9291 & 0.9984 & 0.9235 & 0.9595 & 0.5629 & 0.985 & 0.7164 \\
    \hline
  \end{tabular}%
  \caption{分类指标更新}
  \label{tab:addlabel}%
\end{table}%


\section{下周计划}
\begin{itemize}
\item [1.] [***] 完成技术方案图
\item [2.] [***] 标题-正文匹配更新attention方案
% \item [3.] [***] 调研自监督方法,尝试构建负样本
\end{itemize}
%\bibliography{reference}
%\bibliographystyle{coling}
%\bibliography{coling2020}

\end{CJK}
\end{document}


% include your own bib file like this:


%\begin{thebibliography}{}

%\bibitem[\protect\citename{Aho and Ullman}1972]{Aho:72}
%Alfred~V. Aho and Jeffrey~D. Ullman.
%\newblock 1972.
%\newblock {\em The Theory of Parsing, Translation and Compiling}, volume~1.
%\newblock Prentice-{Hall}, Englewood Cliffs, NJ.

%\bibitem[\protect\citename{{American Psychological Association}}1983]{APA:83}
%{American Psychological Association}.
%\newblock 1983.
%\newblock {\em Publications Manual}.
%\newblock American Psychological Association, Washington, DC.

%\bibitem[\protect\citename{{Association for Computing Machinery}}1983]{ACM:83}
%{Association for Computing Machinery}.
%\newblock 1983.
%\newblock {\em Computing Reviews}, 24(11):503--512.

%\bibitem[\protect\citename{Chandra \bgroup et al.\egroup }1981]{Chandra:81}
%Ashok~K. Chandra, Dexter~C. Kozen, and Larry~J. Stockmeyer.
%\newblock 1981.
%\newblock Alternation.
%\newblock {\em Journal of the Association for Computing Machinery},
%  28(1):114--133.

%\bibitem[\protect\citename{Gusfield}1997]{Gusfield:97}
%Dan Gusfield.
%\newblock 1997.
%\newblock {\em Algorithms on Strings, Trees and Sequences}.
%\newblock Cambridge University Press, Cambridge, UK.

%\bibitem[\protect\citename{Rasooli and Tetreault}2015]{rasooli-tetrault-2015}
%Mohammad~Sadegh Rasooli and Joel~R. Tetreault. 2015.
%\newblock {Yara parser: {A} fast and accurate dependency parser}.
%\newblock \emph{Computing Research Repository}, arXiv:1503.06733.
%\newblock Version 2.

%\bibitem[\protect\citename{Borschinger and Johnson}2011]{borsch2011}
%Benjamin Borschinger and Mark Johnson. 2011.
%\newblock A particle filter algorithm for {B}ayesian wordsegmentation.
%\newblock In \emph{Proceedings of the Australasian Language Technology Association %Workshop 2011}, pages 10--18, Canberra, Australia.

%\end{thebibliography}

