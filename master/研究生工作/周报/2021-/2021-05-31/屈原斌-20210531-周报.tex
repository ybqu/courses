%
% File coling2020.tex
%
% Contact: feiliu@cs.ucf.edu & liang.huang.sh@gmail.com
%% Based on the style files for COLING-2018, which were, in turn,
%% Based on the style files for COLING-2016, which were, in turn,
%% Based on the style files for COLING-2014, which were, in turn,
%% Based on the style files for ACL-2014, which were, in turn,
%% Based on the style files for ACL-2013, which were, in turn,
%% Based on the style files for ACL-2012, which were, in turn,
%% based on the style files for ACL-2011, which were, in turn, 
%% based on the style files for ACL-2010, which were, in turn, 
%% based on the style files for ACL-IJCNLP-2009, which were, in turn,
%% based on the style files for EACL-2009 and IJCNLP-2008...

%% Based on the style files for EACL 2006 by 
%%e.agirre@ehu.es or Sergi.Balari@uab.es
%% and that of ACL 08 by Joakim Nivre and Noah Smith

\documentclass[11pt]{article}
\usepackage{coling2020}
\usepackage{times}
\usepackage{url}
\usepackage{latexsym}
\usepackage{indentfirst}

\usepackage{times}
\usepackage{latexsym}
\usepackage{times}
\usepackage{soul}
\usepackage{url}
\usepackage{amsmath}
\usepackage{amsthm}
\usepackage{booktabs}
\usepackage{algorithm}
\usepackage{algorithmic}
\usepackage{amssymb}
\usepackage{longtable}
\usepackage{graphicx}
\usepackage{CJK}
\usepackage{multirow}
\usepackage{color}

%\setlength\titlebox{5cm}
\colingfinalcopy % Uncomment this line for the final submission

% You can expand the titlebox if you need extra space
% to show all the authors. Please do not make the titlebox
% smaller than 5cm (the original size); we will check this
% in the camera-ready version and ask you to change it back.


\title{周报,2021年05月31日}

\author{屈原斌 \\
  首都师范大学 \\
    {\tt ybqu@cnu.edu.cn}}

\date{}

\begin{document}
\begin{CJK}{UTF8}{gkai}

\maketitle
\CJKindent
%\begin{abstract}

%\end{abstract}
\section{上周计划}


\begin{itemize}
  \item [1.] 通用场景离题检测实验更新
  \item [2.] 标题-正文匹配实验更新
\end{itemize}


\section{上周计划执行情况}
\begin{itemize}
\item [1.] [$\checkmark$]
\item [2.] [$\times$] 
\end{itemize}

\section{本周部分重点工作详述}

\subsection{英文数据集}

\begin{itemize}
  \item 数据集:
  \begin{itemize}
    \item ICLE数据集,共10个主题,离题:不离题=51:733(1:14)
  \end{itemize}
  \item 实验策略:
  \begin{itemize}
    \item 统计正文中包含prompt中关键词的个数,按照分位数划分,分别调参
  \end{itemize}
  \item 指标见表1
  \item 结论:
  \begin{itemize}
    \item TFIDF分布与BERT生成模型结论相反
    \item TFIDF上的结果表明对于不离题作文相对spearman相关系数较高
  \end{itemize}
\end{itemize}

% Table generated by Excel2LaTeX from sheet '中文'
\begin{table}[htbp]
  \centering
  \resizebox*{\textwidth}{!}{
  \begin{tabular}{c|c|ccccccccccc}
    \hline
    \multicolumn{2}{c}{} & \textbf{离题作文数} & \textbf{R@10} & \textbf{R@20} & \textbf{R@50} & \textbf{R@all} & \textbf{P@1} & \textbf{P@5} & \textbf{P@10} & \textbf{spearman} & \textbf{ndcg} & \textbf{ndcg@10} \\
    \hline
    \multirow{4}[0]{*}{\textbf{Bert\_Gen}} & \textbf{$\geq$中位数} & 11    & 0.6833  & 0.6833  & 0.6833  & 0.6833  & 0.3000  & 0.1600  & 0.0800  & 0.2165  & 0.9863  & 0.8991  \\
    \cline{2-13}
    & \textbf{$\textless$中位数} & 40    & 0.6098  & 0.6169  & 0.6383  & 0.6526  & 0.7000  & 0.3400  & 0.2200  & 0.1739  & 0.9697  & 0.8432  \\
    \cline{2-13}
    & \textbf{$\geq$1/4分位数} & 6     & 0.4500  & 0.5000  & 0.5000  & 0.5000  & 0.0000  & 0.0800  & 0.0500  & 0.1834  & 0.9870  & 0.9300  \\
    \cline{2-13}
    & \textbf{$\textless$1/4分位数} & 45    & 0.5181  & 0.5681  & 0.5748  & 0.5748  & 0.6000  & 0.3600  & 0.2200  & 0.0234  & 0.9665  & 0.8540  \\
    \hline
    \multirow{4}[0]{*}{\textbf{TFIDF}} & \textbf{$\geq$中位数} & 11    & 0.4000  & 0.4000  & 0.5333  & 0.6000  & 0.1000  & 0.0600  & 0.0400  & 0.0890  & 0.9808  & 0.9193  \\
    \cline{2-13}
    & \textbf{$\textless$中位数} & 40    & 0.5417  & 0.7036  & 0.7583  & 0.7655  & 0.6000  & 0.3800  & 0.2300  & 0.1813  & 0.9652  & 0.8340  \\
    \cline{2-13}
    & \textbf{$\geq$1/4分位数} & 6     & 0.3500  & 0.3500  & 0.4000  & 0.4000  & 0.1000  & 0.0800  & 0.0400  & 0.1715  & 0.8872  & 0.8354  \\
    \cline{2-13}
    & \textbf{$\textless$1/4分位数} & 45    & 0.4374  & 0.5640  & 0.6440  & 0.6774  & 0.5000  & 0.3600  & 0.2300  & 0.1475  & 0.9678  & 0.8364  \\
    \hline
  \end{tabular}}%
  \caption{英文数据集实验结果}
  \label{tab:addlabel}%
\end{table}%


\subsection{中文数据集}
\begin{itemize}
  \item 数据集:
  \begin{itemize}
    \item 共2100篇测试作文,离题:不离题=350:1750(1:5)
  \end{itemize}
  \item 实验策略:
  \begin{itemize}
    \item 直接在测试集上调参
  \end{itemize}
  \item 指标见表2
  \item 结论:
  \begin{itemize}
    \item 增大离题比例后,指标整体下降
    \item TFIDF指标较分布式表示差(结果分析)
  \end{itemize}
\end{itemize}

% Table generated by Excel2LaTeX from sheet '中文'
\begin{table}[htbp]
  \centering
  \resizebox*{\textwidth}{!}{
  \begin{tabular}{c|ccccccccccc}
    \hline
    & \textbf{R@10} & \textbf{R@15} & \textbf{R@20} & \textbf{R@50} & \textbf{R@all} & \textbf{P@1} & \textbf{P@5} & \textbf{P@10} & \textbf{spearman} & \textbf{ndcg} & \textbf{ndcg@10} \\
    \hline
    \textbf{TFIDF} & 0.1703  & 0.2503  & 0.3243  & 0.4781  & 0.4781  & 0.7000  & 0.7600  & 0.6200  & 0.0113  & 0.8771  & 0.5880  \\
    \hline
    \textbf{Skip-gram} & 0.0799  & 0.0986  & 0.1137  & 0.2364  & 0.6238  & 0.7000  & 0.4600  & 0.2700  & -0.0005  & 0.9678  & 0.8037  \\
    \hline
    \textbf{Doc2vec} & 0.0870  & 0.1293  & 0.1715  & 0.3027  & 0.5293  & 0.7000  & 0.5200  & 0.3100  & 0.0647  & 0.9794  & 0.7532  \\
    \hline
    \textbf{Bert\_Classification} & 0.1438  & 0.1599  & 0.1822  & 0.2980  & 0.7782  & 0.5000  & 0.5200  & 0.5000  & 0.1479  & 0.9799  & 0.8257  \\
    \hline
    \textbf{Bert-Gen} & 0.1677  & 0.1810  & 0.2030  & 0.2707  & 0.7145  & 0.5000  & 0.5200  & 0.5900  & 0.1219  & 0.9638  & 0.8527  \\
    \hline
  \end{tabular}}%
  \caption{中文数据集实验结果}
  \label{tab:addlabel}%
\end{table}%


\section{下周计划}
\begin{itemize}
\item [1.] [***] 实验结果分析
\item [2.] [***] 标题-正文匹配测试集check,更新实验结果
\item [3.] [****] 尝试更新一些HT表示(加入关键字)
\end{itemize}
%\bibliography{reference}
%\bibliographystyle{coling}
%\bibliography{coling2020}

\end{CJK}
\end{document}


% include your own bib file like this:


%\begin{thebibliography}{}

%\bibitem[\protect\citename{Aho and Ullman}1972]{Aho:72}
%Alfred~V. Aho and Jeffrey~D. Ullman.
%\newblock 1972.
%\newblock {\em The Theory of Parsing, Translation and Compiling}, volume~1.
%\newblock Prentice-{Hall}, Englewood Cliffs, NJ.

%\bibitem[\protect\citename{{American Psychological Association}}1983]{APA:83}
%{American Psychological Association}.
%\newblock 1983.
%\newblock {\em Publications Manual}.
%\newblock American Psychological Association, Washington, DC.

%\bibitem[\protect\citename{{Association for Computing Machinery}}1983]{ACM:83}
%{Association for Computing Machinery}.
%\newblock 1983.
%\newblock {\em Computing Reviews}, 24(11):503--512.

%\bibitem[\protect\citename{Chandra \bgroup et al.\egroup }1981]{Chandra:81}
%Ashok~K. Chandra, Dexter~C. Kozen, and Larry~J. Stockmeyer.
%\newblock 1981.
%\newblock Alternation.
%\newblock {\em Journal of the Association for Computing Machinery},
%  28(1):114--133.

%\bibitem[\protect\citename{Gusfield}1997]{Gusfield:97}
%Dan Gusfield.
%\newblock 1997.
%\newblock {\em Algorithms on Strings, Trees and Sequences}.
%\newblock Cambridge University Press, Cambridge, UK.

%\bibitem[\protect\citename{Rasooli and Tetreault}2015]{rasooli-tetrault-2015}
%Mohammad~Sadegh Rasooli and Joel~R. Tetreault. 2015.
%\newblock {Yara parser: {A} fast and accurate dependency parser}.
%\newblock \emph{Computing Research Repository}, arXiv:1503.06733.
%\newblock Version 2.

%\bibitem[\protect\citename{Borschinger and Johnson}2011]{borsch2011}
%Benjamin Borschinger and Mark Johnson. 2011.
%\newblock A particle filter algorithm for {B}ayesian wordsegmentation.
%\newblock In \emph{Proceedings of the Australasian Language Technology Association %Workshop 2011}, pages 10--18, Canberra, Australia.

%\end{thebibliography}

